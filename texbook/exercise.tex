%% \tracingall
\input manmac
\hsize=6.5in
\vsize=8.9in
\def\myhr{\vskip.5\baselineskip\hrule\vskip.5\baselineskip}


\noindent Chapter 1
\myhr
\ansno1.1:
 A \TeX nician (underpaid); sometimes also called a \TeX acker.
\myhr
\vfill
\eject


\noindent Chapter 2

\myhr
%% [2025-05-06T18:14:46+08:00]
``I understand''

%% [2025-05-06T18:12:00+08:00] Do not write comment inside the block
%% between `\begintt' and `\endtt' because the anything between
%% `\begintt' and `\endtt' will be eventually printed out in output
%% pdf.

%% [2025-05-06T18:16:22+08:00] `\]' represents "visible spacebar" in
%% normal/display mode
Visible spacebar: ``I\]understand.''

%% [2025-05-06T18:13:44+08:00] `|]' represents "visible spacebar" in
%% verbatim mode
\begintt
Verbatim spacebar: ``I|]understand.''
\endtt

\begindisplay
|-| & hyphen (-)\cr
|--| & en-dash (--)\cr
|---| & em-dash (---)\cr
|$-$| & minus sign $-$\cr
\enddisplay
\hrule

\ansno2.1: Alice said, ``I always use an en-dash instead of a hyphen
when specifying page numbers like `480--491' in a bibliography.''


\ansno2.2: Four hyphens `|----|' becomes one em-dash and one hyphen
`----'

\ansno2.3: fluff
\myhr
\noindent{\bf Summary}:
\item\bull ligatures
\item\bull kerning
\item\bull alternative input methods for left quotes, right quotes

\begindisplay
{\tt |\lq\lq\]I\]understand.\rq\rq|} $\Rightarrow$ \lq\lq\]I\]understand.\rq\rq
\enddisplay
\hrule


\ansno2.4: ``\thinspace` and `\thinspace``


\ansno2.5: Following Occam's Razor Principle, introducing extra
complexity on typewriting annotations might not be a wise move because
it increases extra maintenance cost in minds and memory. Besides,
introducing dollar sign might cause confusions between {\sl Math mode}
and {\sl Text mode} that it may deteriorate maintenability of code.
\myhr
\vfill
\eject


\noindent Chapter 3.
\myhr

Note: |ESC| key $\neq$ escape character

\noindent{\bf Summary}:
\item\bull control sequence, control symbol

\myhr
\ansno3.1:
`|\I'm \exercise3.1\\!|' control sequences are `|\I|', `|\exercise|',
 and `|\\|'

\ansno3.2:
\begindisplay
`math\'ematique' & $\rightarrow$\quad |`math\'ematique'|\cr
`centim\`etre' & $\rightarrow$\quad |`centim\`etre'|\cr
\enddisplay
\myhr
note: \TeX\ ignores spaces after control worlds

\noindent{\bf Summary}:
\item\bull apprimately 300 of 900 control sequence in \TeX\ are {\sl primitive}
\myhr
\ansno3.3: |\|\] is primitive, while |\|\<return> is represented as |^^M|

\ansno3.4:
([2025-05-09T00:11:26+08:00] after asking Perplexity.AI about how to
understand Knuth's answer, it makes sense now)

For length~2 case, the first position is always occupied by `|\|', the
remaining position is for a single ASCII character, which is 256
different variants (in the problem description, it said `including
escape character').

For length~3, it might be strictly limited to alphabet characters
(i.e. |[A-Za-z]| if represented in Regex). The first position is
contantly as `|\|', and each of the remaining positions has 52
variants, which is $52^2$ possible control sequences.
\myhr

\vfill
\eject


\noindent Chapter 4.
\myhr
\begindisplay
|\rm| & {\rm Roman}\cr
|\sl| & {\sl Slanted}\cr
|\it| & {\it Italic}\cr
|\tt| & {\tt Typewriter}\cr
|\bf| & {\bf Bold}\cr
\enddisplay
\myhr

\ansno4.1:
`Ulrich Dieter, {\sl Journal f\"ur die reine und angewandte
Mathematik} {\bf 201} (1959), 37--70.'

\ansno4.2:
{\it Explain how to typeset a\/ {\rm roman} word in the midst of an
italicized sentence.}

\ansno4.3:
|\def\ic#1{\setbox0=\hbox{#1\/}\dimen0=\wd0|\parbreak
|\setbox0=\hbox{#1}\advance\dimen0 by -\wd0 }|.
\myhr
\tenrm smaller \ninerm and smaller
\eightrm and smaller \sevenrm and smaller
\sixrm and smaller \fiverm and smaller \tenrm
\myhr
\ansno4.4:
The numeric ones would be possibly interpreted as `|\1|' and `0point'
instead of the entire control sequence.

\ansno4.5:
Quoted from official answer: ``Say |\def\sl{\it}| at the beginning,
and delete other definitions of\/ |\sl| that might be present in your
format file (e.g., there might be one inside a |\tenpoint| macro).''

%% \myhr

%% [2025-05-12T09:43:23+08:00] If it's ttf/otf font, it is required to
%% be preprocessed by mktexmf first, otherwise an error will be thrown
%% out
%%
%%     kpathsea: Running mktextfm Cambria
%%
%%     The command name is $TEXLIVE_HOME\2024\bin\windows\mktextfm
%%
%%     kpathsea: Running mktexmf Cambria.mf
%%
%%     The command name is $TEXLIVE_HOME\2024\bin\windows\mktexmf
%%     name = Cambria, rootname = Cambria, pointsize =
%%     mktexmf: empty or non-existent rootfile!
%%     Cannot find Cambria.mf.
%%     kpathsea: Appending font creation commands to build/missfont.log.
%%
%% ----
%% \font\ninerm=Cambria at 10pt

%% \myhr

\ansno4.6:
|\font\tenrm=cmr10 at 5pt|\parbreak
|\font\tenrm=cmr10 scaled 500|

Note: In the official answer, it uses `|\squinttenrm|'
\myhr
\font\msb=msbm10 at 10pt
\def\ldots{\mathinner{\ldotp\ldotp\ldotp}}

\noindent{\bf Summary}:
\item\bull magnification ratio factor
\item\bull $1.2^n, n \in $\thinspace {\msb N}
\item\bull |\magstep0|, |\magstep1|, $\ldots$, |\magstep5|
\myhr
\vfill
\eject


\noindent Chapter 5.
\myhr
\noindent{\bf Summary}:
\item\bull Use `|{|', `|}|' for grouping
\myhr
\ansno5.1:
Use grouping: `|shelf{f}ul|'\ $\rightarrow$\ `shelf{f}ul' rather than `shelfful'\parbreak

Official answer: |{shelf}ful| or |shelf{}ful|, etc.; or even
|shelf\/ful|, which yields a shelf\/ful instead of a
shelf{\kern0pt}ful.

\ansno5.2:
Use grouping: `|{\]}{\]}{\]}|' $\rightarrow$ `{\]}{\]}{\]}'\parbreak

Official answer: `\]|{|\]|}|\]' or `\]|{}|\]|{}|\]', etc. Plain \TeX\
also has a |\space| macro, so you can type |\space\space\space|.  \
(These aren't strictly equivalent to `|\|\]|\|\]|\|\]', since they
adjust the spaces by the current ``{space factor},'' as explained
later.)
\myhr
\noindent{\bf Summary}:

\item\bull |\centerline|
\item\bull grouping
\item\bull scope
\item\bull nested

Example:
\begindisplay
\centerline{This information should be {\it centered}.}
\enddisplay
\myhr
\ansno5.3:
First case: same\parbreak
second case: Only `S' centred but rest would be automatically arranged.

\ansno5.4:
The line will be centred and `centered' will be italic.

\ansno5.5:
\def\ital#1{{\it #1\/}}\parbreak
|\def\ital#1{{\it #1\/}}|

|\ital{hello} world| $\rightarrow$ \ital{hello} world

\item\bull Pros: scoped, more legible
\item\bull Cons: learning cost

%% \global\advance\count0 by 1

\ansno5.6:
Official answer: |{1 {2 3 4 5} 4 6} 4|

%% \def\c#1{\count1=#1}
%% \def\g{\global\count1=}
%% \def\s{\showthe\count1}

%% \hbox{%
%%   \c1\s
%%   \g2
%%   {%
%%     \s
%%     \c3\s
%%     \g4\s
%%     \c5\s
%%   }%
%%   \s
%%   \c6\s
%% }%
%% \s

%% ===================================================================
%% [2025-05-13T19:40:35+08:00] Cannot reproduce the result neither in
%% TeX nor in pdfTeX. It keeps alarming errors like
%%
%%     <to be read again>
%%                        \global
%%     \g ->\global
%%                  \count 1=
%%     l.268   \g
%%               2

%% it does not work even braces added/removed or marco parameter
%% added/removed. It seems like the behaviorial differences between
%% different versions of TeX.
\myhr
\vfill
\eject

\noindent Chapter 6.
\myhr
$$\halign{\hbox to\parindent{\hfil\sevenrm#\ \ }&#\hfil\cr
1&|\hrule|\cr
2&|\vskip 1in|\cr
3&|\centerline{\bf A SHORT STORY}|\cr
4&|\vskip 6pt|\cr
5&|\centerline{\sl by A. U. Thor}|\cr
6&|\vskip .5cm|\cr
7&|Once upon a time, in a distant|\cr
8&|  galaxy called \"O\"o\c c,|\cr
9&|there lived a computer|\cr
10&|named R.~J. Drofnats.|\cr
11&||\cr
12&|Mr.~Drofnats---or ``R. J.,'' as|\cr
13&|he preferred to be called---|\cr
14&|was happiest when he was at work|\cr
15&|typesetting beautiful documents.|\cr
16&|\vskip 1in|\cr
17&|\hrule|\cr
18&|\vfill\eject|\cr}$$

\vskip.5\baselineskip

\hrule
\vskip 1in
\centerline{\bf A SHORT STORY}
\vskip 6pt
\centerline{\sl by A. U. Thor}
\vskip .5cm
Once upon a time, in a distant
  galaxy called \"O\"o\c c,
there lived a computer
named R.~J. Drofnats.

Mr.~Drofnats---or ``R. J.,'' as
he preferred to be called---
was happiest when he was at work
typesetting beautiful documents.
\vskip 1in
\hrule
\vfill\eject
\myhr
\noindent{\bf Summary}:

\item\bull `|\c|' (cedilla)
\item\bull `|~|' (ties, treat as normal space but not
to break between lines, e.g. names)
\myhr
\ansno6.1:
Official answer: ``Laziness and/or obstinacy''

\ansno6.2:
Official answer: `` There's an unwanted space after `called---',
because (as the book says) \TeX\ treats the end of a line as if it
were a blank space. That blank space is usually what you want, except
when a line ends with a hyphen or a dash; so you should WATCH OUT for
lines that end with hyphens or dashes''

\ansno6.3:
Official answer: ``It represents the heavy bar that shows up in your
output. \ (This bar wouldn't be present if had been set to |0pt|, nor
is it present in an underfull box.)''

\ansno6.4:
Official answer: ``This is the |\parfillskip| space that ends the
paragraph.  In plain \TeX\ the parfillskip is zero when the last line
of the paragraph is full; hence no space actually appears before the
rule in the output of Experiment~3. But all hskips show up as spaces
in an overfull box message, even if they're zero.''

\ansno6.5:
Official answer: ``Run \TeX\ with \hbox{|\hsize=1.5in|} \hbox{|\tolerance=10000|}
\hbox{|\raggedright|} \hbox{|\hbadness=-1|} and then |\input story|. \TeX\ will
report the badness of all lines (except the final lines of paragraphs,
where fill glue makes the badness zero).''

\ansno6.6:
Official answer: ``|\def\extraspace{\nobreak \hskip 0pt plus
.15em\relax}|\parbreak
|\def\dash{\unskip\extraspace---\extraspace}|\par\nobreak\smallskip\noindent
(If you try this with the story at 2-inch and 1.5-inch sizes, you will
notice a substantial improvement. The |\unskip| allows people to leave
a space before typing |\dash|.  \TeX\ will try to hyphenate before
|\dash|, but not before `|---|'; cf.\ Appendix~H\null. The |\relax|
at the end of |\extraspace| is a precaution in case the next word is
`|minus|'.)''

\ansno6.7:
Official answer: ``\TeX\ would have deleted five tokens: |1|, |i|,
|n|, \], |\centerline|.  (The space was at the end of line~2, the
|\centerline| at the beginning of line~3.)''

\ansno6.8:
Official answer: ``A control sequence like |\centerline| might well
define a control sequence like |\ERROR| before telling \TeX\ to look
at |#1|. Therefore
\TeX\ doesn't interpret control sequences when it scans an argument.''

\myhr
\vfill
\eject


\noindent Chapter 7.
\myhr

\noindent{\bf Summary}:
\item\bull Reserved $10$ characters (must be escaped with `|\|'): `|\|', `|{|', `|}|', `|$|', `|&|', `|#|', `|^|', `|_|', `|%|', `|~|'
\item\bull example: `|$\{a \backslash b\}$|' $\rightarrow$ `$\{a \backslash b\}$'
\myhr
\ansno7.1:
The `|&|', `|$|' and `|%|' must be escaped.

\ansno7.2:
Official answer: ``Reverse slashes (backslashes) are fairly uncommon
in formulas or text, and |\\| is very easy to type; it was therefore
felt best not to reserve |\\| for such limited use. Typists can define
|\\| to be whatever they want (including |\backslash|).''
\myhr
\noindent{\bf Summary}:
\item\bull \cstok{boxed token}
\myhr
\ansno7.3:
Official answer: ``1, 2, 3, 4, 6, 7, 8, 10, 11, 12, 13. {Active
characters} (type 13) are somewhat special; they behave like control
sequences in most cases (e.g., when you say `|\let||\x=~|' or
`|\ifx||\x~|'), but they behave like character tokens when they appear
in the token list of\/ |\uppercase| or |\lowercase|, and when
unexpanded after |\if| or |\ifcat|.''

\ansno7.4:
Official answer: ``It ends with either |>| or |}| or any character of
category 2; then the effects of all |\catcode| definitions within the
group are wiped out, except those that were |\global|.''

\ansno7.5:
Official answer: ``If you type `|\message{\string~}|' and `|\message{\string\~}|', \TeX\
responds with `|~|' and `|\~|', respectively. |\message|
To get |\|$_{12}$ from |\string| you therefore need to make backslash an
active character. One way to do this is
\begintt
{\catcode`/=0 \catcode`\\=13 /message{/string\}}
\endtt
(The ``{null control sequence}'' that you get when there are no
tokens between |\csname| and |\endcsname| is not a solution to this exercise,
because |\string| converts it to `|\csname\endcsname|'. There is, however,
another solution: If \TeX's |\escapechar| parameter---which will be
explained in one of the next dangerous bends---is negative or greater
than~255, then `|\string\\|' works.)''

\ansno7.6:
|\|$_{12}$ |a|$_{12}$ |\|$_{12}$ \]$_{10}$ |b|$_{12}$.

\ansno7.7:
Official answer: ``|\def\ifundefined#1{\expandafter\ifx\csname#1\endcsname\relax}|%
\hfil\break Note that a control sequence like this must be used with care;
it cannot be included in {conditional} text, because the |\ifx| will not
be seen when |\ifundefined| isn't expanded.''

\ansno7.8:
|\uppercase{a\lowercase{bC}}| $\rightarrow$ `\uppercase{a\lowercase{bC}}'

\ansno7.9:
`\copyright\ \uppercase\expandafter{\romannumeral\year}' $\rightarrow$ `|\copyright\ \uppercase\expandafter{\romannumeral\year}|'

\ansno7.10:
Official answer:
\begintt
\def\gobble#1{} % remove one token
\def\appendroman#1#2#3{\expandafter\def\expandafter#1\expandafter
  {\csname\expandafter\gobble\string#2\romannumeral#3\endcsname}}
\endtt
\myhr
\vfill
\eject


\noindent Chapter 8.
\myhr
\noindent{\bf Summary}:

\item\bull `|\char98 u\char98\char98 le|' $\rightarrow$ `\char98 u\char98\char98 le'

\halign{\indent\hfil# && $\rightarrow$ #\cr
`|\char98|' & `\char98'\cr
`|\char'142|' & `\char'142'\cr
`|\char"62|' & `\char"62'\cr
}

\myhr
\ansno8.1:

|%| in \TeX can be interpreted as reserved keyword of comment
|%| character.

Official answer: ``The |%| would be treated as a comment character,
because its category code is~14; thus, no |%| token or |}| token would
get through to the gullet of \TeX\ where numbers are treated. When a
character is of category 0, 5, 9, 14, or~15, the extra |\| must be
used; and the |\| doesn't hurt, so you can always use it to be safe.''

\ansno8.2:
\halign{\indent\hfil#\ &\ #\hfil\cr
(a)&A caracter of category~5 can be converted into `\]'$_{10}$ or
a \cstok{par} token, \cr
&while a character of category~14 never produces a token.\cr
(b)&Different category numbers\cr
(c)&Same as (b)\cr
(d)&No\cr
(e)&Yes\cr
(f)&No\cr
}
\myhr
\ansno8.3: TBD
\ansno8.4: TBD
\ansno8.5: TBD
\ansno8.6: TBD
\ansno8.7: TBD

\myhr
\vfill
\eject


\noindent Chapter 9.
\myhr

\noindent{\bf Summary}:
\item\bull upper accent

$$\halign{\indent\hbox to 50pt{#\hfil}&\hbox to 35pt{#\hfil}&#\hfil\cr
\it\negthinspace Type&\it to get\cr
\noalign{\smallskip}
|\`o|&\`o&(grave accent)\cr
|\'o|&\'o&(acute accent)\cr
|\^o|&\^o&(circumflex or ``hat'')\cr
|\"o|&\"o&(umlaut or dieresis)\cr
|\~o|&\~o&(tilde or ``squiggle'')\cr
|\=o|&\=o&(macron or ``bar'')\cr
|\.o|&\.o&(dot accent)\cr
|\u o|&\u o&(breve accent)\cr
|\v o|&\v o&(h\'a\v cek or ``check'')\cr
|\H o|&\H o&(long Hungarian umlaut)\cr
|\t oo|&\t oo&(tie-after accent)\cr}$$

accents that go underneath

$$\halign{\indent\hbox to 50pt{#\hfil}&\hbox to 35pt{#\hfil}&#\hfil\cr
\it\negthinspace Type&\it to get\cr
\noalign{\smallskip}
|\c o|&\c o&(cedilla accent)\cr
|\d o|&\d o&(dot-under accent)\cr
|\b o|&\b o&(bar-under accent)\cr}$$

Special letters

$$\halign{\indent\hbox to 50pt{#\hfil}&\hbox to 35pt{#\hfil}&#\hfil\cr
\it\negthinspace Type&\it to get\cr
\noalign{\smallskip}
|\oe,\OE|&\oe,\thinspace\OE&(French ligature OE)\cr
|\ae,\AE|&\ae,\thinspace\AE&(Latin ligature and Scandinavian letter AE)\cr
|\aa,\AA|&\aa,\thinspace\AA&(Scandinavian A-with-circle)\cr
|\o,\O|&\o,\thinspace\O&(Scandinavian O-with-slash)\cr
|\l,\L|&\l,\thinspace\L&(Polish suppressed-L)\cr
|\ss|&\ss&(German ``es-zet'' or sharp S)\cr}$$
\myhr
\ansno9.1: `|na\"\i ve|' $\rightarrow$ `na\"\i ve'

\ansno9.2:
Official answer: ``Belov\`ed prot\'eg\'e; r\^ole co\"ordinator;
souffl\'es, cr\^epes, p\^at\'es, etc.''

\ansno9.3:
`|\AE sop's \OE uvres en fran\c cais|' $\rightarrow$ `\AE sop's \OE
uvres en fran\c cais'

\ansno9.4:
`|{\sl Commentarii Academi\ae\ scientiarum imperialis|\hfil\break
|petropolitan\ae\/} became {\sl Akademi\t\i a Nauk SSSR,| \hfil\break
|Doklady}|' $\rightarrow$ `{\sl Commentarii Academi\ae\ scientiarum
imperialis petropolitan\ae\/} became {\sl Akademi\t\i a Nauk SSSR,
Doklady}'

\ansno9.5:
\halign{\indent\hfil#$\rightarrow$&\ #\hfil\cr
`|Ernesto {Ces\`aro}|'                           & `Ernesto {Ces\`aro}'\cr
`|P\'al {Erd\H os}|'                             & `P\'al {Erd\H os}'\cr
`|\O ystein {Ore}|'                              & `\O ystein {Ore}'\cr
`|Stanis\l aw \'Swierczkowski|'                  & `Stanis\l aw \'Swierczkowski' \cr
`|Serge\u\i\ \t Iur'ev|'                         & `Serge\u\i\ \t Iur'ev'\cr
`|Mu\d hammad ibn M\^us\^a {al-Khw\^arizm\^\i}|' & `Mu\d hammad ibn M\^us\^a {al-Khw\^arizm\^\i}'\cr
}

\ansno9.6:
`|{\tt P\'al Erd{\bf\H{\tt o}}s}|' $\rightarrow$ `{\tt P\'al Erd{\bf\H{\tt o}}s}'
\myhr
\noindent{\bf Summary}:
\item\bull Special signs

$$\halign{\indent#\hfil\ &\hfil#\hfil&#\hfil\cr
\it\negthinspace Type&\it to get\cr
\noalign{\smallskip}
|\dag|&\dag&(dagger or obelisk)\cr
|\ddag|&\ddag&(double dagger or diesis)\cr
|\S|&\S&(section number sign)\cr
|\P|&\P&(paragraph sign or pilcrow)\cr}$$
\myhr
\ansno9.7:
`|{\it Europe on {\sl\$}15.00 a day\/}|' $\rightarrow$ `{\it Europe on
{\sl\$}15.00 a day\/}'

\ansno9.8:
Official answer: ``The extra braces keep font changes local. An
argument makes the use of\/ |\'| more consistent with the use of other
accents like |\d|, which are manufactured from other characters
without using the |\accent| primitive.''
\myhr
\vfill
\eject


\noindent Chapter 10.

\myhr
\noindent{\bf Summary}
\item\bull Units

$$\halign{\indent\tt#&\quad#\hfil\cr
pt&point (baselines in this manual are $12\pt$ apart)\cr
pc&pica ($\rm1\,pc=12\,pt$)\cr
in&inch ($\rm1\,in=72.27\,pt$)\cr
bp&big point ($\rm72\,bp=1\,in$)\cr
cm&centimeter ($\rm2.54\,cm=1\,in$)\cr
mm&millimeter ($\rm10\,mm=1\,cm$)\cr
dd&didot point ($\rm1157\,dd=1238\,pt$)\cr
cc&cicero ($\rm1\,cc=12\,dd$)\cr
sp&scaled point ($\rm65536\,sp=1\,pt$)\cr}$$
\myhr
\ansno10.1:
254 centimeters $=$ 100 in $=$ 7227 pt
\myhr
\noindent{\bf Summary}:
\item\bull Parsing syntax

$$\halign{\indent\hfil#&&\quad #\hfil\cr
Syntax 1:&\<optional sign>\<number>\<unit of measure>\cr
Syntax 2:&\<optional sign>\<digit string>|.|\<digit string>\<unit of measure>\cr
}$$
\myhr
\ansno10.2:
\halign{\indent\hfil#\ &\hfil#&\thinspace #\hfil\ &\ #\cr
1 & -.013837  & in & (-0.3514598 mm)\cr
2 & 0.        & mm & (0.0 mm)\cr
3 & +42.1     & dd & (15.832338 mm)\cr
4 & 3         & in & (76.2 mm)\cr
5 & 29        & pc & (122.308012 mm)\cr
6 & 123456789 & sp & (662.080365 mm)\cr
}

\ansno10.3:
\halign{\indent\hfil#\ &\ #\hfil\cr
|'.77pt| & Not legitimate \cr
|"Ccc| & $\rightarrow 12$ cc\cr
|-,sp| & $\rightarrow 0.0$ sp\cr
}

\ansno10.4:
Official answer: ``
% tc: tianchen, tpf: ten point four, bs: backslash
\begintt
\def\tick#1{\vrule height 0pt depth #1pt}
\def\tctpfbs{\hbox to 1cm{\hfil\tick4\hfil\tick8}}
\vbox{\hrule\hbox{\tick8\tctpfbs\tctpfbs\tctpfbs\tctpfbs
\tctpfbs\tctpfbs\tctpfbs\tctpfbs\tctpfbs\tctpfbs}}
\endtt''

\indent\vbox{
\def\tick#1{\vrule height 0pt depth #1pt}
\def\tctpfbs{\hbox to 1cm{\hfil\tick4\hfil\tick8}}
\vbox{\hrule\hbox{\tick8\tctpfbs\tctpfbs\tctpfbs\tctpfbs
\tctpfbs\tctpfbs\tctpfbs\tctpfbs\tctpfbs\tctpfbs}}
}
\myhr
\noindent{\bf Summary}:

\halign{\indent\hfil#\ &\ #\hfil\cr
|\magnification=1200| & Enlarge 1.2 times normal size\cr
|\magnification=2000| & Double\cr
}
\myhr
\ansno10.5:

\vbox{
1.2 times\vskip5pt
\font\enlargedrm=cmr10 scaled\magstep1
\enlargedrm\input story
}
\vbox{
\vskip5pt 1.44 times\vskip5pt
\font\enlargedrm=cmr10 scaled\magstep2
\enlargedrm\input story
}
\vbox{
\vskip5pt 1.728 times\vskip5pt
\font\enlargedrm=cmr10 scaled\magstep3
\enlargedrm\input story
}
\myhr

\noindent{\bf Summary}:
\item\bull |\hsize=6.5 true in| represents no-scaled 6.5 inch for text area height

\myhr
\ansno10.6:
%% [2025-05-22T17:27:24+08:00] `\magnification' Not work in pdftex
Official answer: ``Magnification is by a factor of 1.2. Since font |\first| is |cmr10|
at $12\pt$, it will be |cmr10| at $14.4\pt$ after magnification;
font |\second| will be |cmr10| at $12\pt$. \ (\TeX\ changes
`|12truept|' into `|10pt|', and the final output magnifies it back to
$12\pt$.)''
\myhr
\vfill
\eject


\noindent Chapter 11.
\myhr
\noindent{\bf Summary}:
\item\bull black box: `\vrule width 4pt height 6pt depth 1.5pt', `\bull'

\vbox{\hbox{Two lines}\hbox{of type.}}
\myhr

\ansno11.1:
The `|E|' is inside a box that is inside a box.

\ansno11.2:
Official answer: ``The idea is to construct a box and to look inside. For example,
\begintt
\setbox0=\hbox{\sl g\/} \showbox0
\endtt
reveals that |\/| is implemented by placing a kern after the character.
Further experiment shows that this kern is inserted even when the italic
correction is zero.''

\ansno11.3:
Official answer: ``The height, depth, and width of the enclosing box should be just large
enough to enclose all of the contents, so the result is:
\begintt
\hbox(8.98608+0.0)x24.44484
.\tenrm T
.\kern 1.66702
.\hbox(6.83331+0.0)x6.80557, shifted -2.15277
..\tenrm E
.\kern 1.25
.\tenrm X
\endtt
(You probably predicted a height of |8.9861|; \TeX's internal calculations are
in |sp|, not |pt|/100000, so the rounding in the fifth decimal place is not
readily predictable.)''

\ansno11.4:
\TeX\ is supposed to be designed for English-language world. For a general,
symmetrical box design, {\it CSS3 Box Model} might be a fair implementation
for reference.

\ansno11.5:
\vbox{
\def\dolist{\afterassignment\dodolist\let\next= }
\def\dodolist{\ifx\next\endlist \let\next\relax
  \else \\\let\next\dolist \fi
  \next}
\def\endlist{\endlist}
\def\hidehrule#1#2{\kern-#1%
  \hrule height#1 depth#2 \kern-#2 }
\def\hidevrule#1#2{\kern-#1{\dimen0=#1
    \advance\dimen0 by#2\vrule width\dimen0}\kern-#2 }
\def\makeblankbox#1#2{\hbox{\lower\dp0\vbox{\hidehrule{#1}{#2}%
    \kern-#1 % overlap the rules at the corners
    \hbox to \wd0{\hidevrule{#1}{#2}%
      \raise\ht0\vbox to #1{}% set the vrule height
      \lower\dp0\vtop to #1{}% set the vrule depth
      \hfil\hidevrule{#2}{#1}}%
    \kern-#1\hidehrule{#2}{#1}}}}
\def\maketypebox{\makeblankbox{0pt}{1pt}}
\def\makelightbox{\makeblankbox{.2pt}{.2pt}}
\def\\{\if\space\next\ % assume that \next is unexpandable
 \else \setbox0=\hbox{\next}\maketypebox\fi}
\def\demobox#1{\setbox0=\hbox{\dolist#1\endlist}%
  \leavevmode\copy0\kern-\wd0\makelightbox}


\noindent|\demobox{Tough exercise.}| $\rightarrow$ \demobox{Tough exercise.}
}

\ansno11.6:
\vbox{
\def\frac#1/#2{\leavevmode\kern.1em
  \raise.5ex\hbox{\the\scriptfont0 #1}\kern-.1em
  /\kern-.15em
  \lower.25ex\hbox{\the\scriptfont0 #2}}

\noindent|\frac1/2| $\rightarrow$ `\frac1/2'
}

\myhr
\vfill
\eject


\noindent Chapter 12.
\myhr
\noindent{\bf Summary}:
\item\bull box with glue example

\begindisplay\eightpoint
\vbox{
    \hbox{\samplebox{7mm}{8mm}{5\varunit}{width 5}%
      \sampleglue{9\varunit}{space 9\cr stretch 3\cr shrink 1}%
      \samplebox{3mm}{2mm}{6\varunit}{width 6}%
      \sampleglue{9\varunit}{space 9\cr stretch 6\cr shrink 2}%
      \samplebox{8mm}{3mm}{3\varunit}{width 3}%
      \sampleglue{12\varunit}{space 12\cr stretch 0\cr shrink 0}%
      \samplebox{4mm}{7mm}{8\varunit}{width 8}}
    \kern6pt
    \arrows{52\varunit}{width 52}}
\enddisplay

\item\bull glue stretch/shrink arithmetics

In the example above, if the total width is expected to be $58$, then
\vskip\baselineskip
\halign{\indent\hfil#\ &\ #\hfil\cr
1 & Calculate how many units are require to be stretched/shrinked\cr
& i.e. $58-52=6$\cr
2 & formula: natural $\pm$ (total actual needed $\div$ total capability $\times$ actual needed)\cr
& i.e. $9 + (6/9)\times 3 = 11$, $9 + (6/9)\times 6 = 13$, $12 + (6/9)\times 0 = 12$\cr
& i.e. $5 + 11 + 6 + 13 + 3 + 12 + 8 = 58$\cr
}

\item\bull setting the glue (once set the glue becomes rigid)
\item\bull glue cannot shrink more than shrinkability, but can stretch arbitraily far

\myhr
\ansno12.1:
\halign{\indent\hfil#\ &\ #\hfil\cr
Step 1.&$100-52=48$ globs require to be added/stretched\cr
Step 2.&$9 + (48/9)\times 3 = 9+16$, $9 + (48/9)\times 6 = 9+32$, $9 + (48/9)\times 0 = 12+0$\cr
}

Since $100$ exceeds the maximal stretchable width, \TeX\ will consider
as ``infinitely bad''.
\myhr
\noindent{\bf Summary}:
\item\bull glue syntax: |\<glue_control_sequence><dimen> plus<dimen> minus<dimen>|
\item\bull |\vfil| vs |\vfill|
\vskip1em
\halign{\indent\hfil#\ &\ #\hfil\cr
|\vfil| & Expand to fill the remaining space\cr
|\vfill| & effectively prevents |\vfil| from stretching\cr
|\hfil| & horizontal fill\cr
|\hfill| & horizontal fill (stronger)\cr
|\hss| & horizontal stretch or shrink\cr
|\vss| & vertical stretch or shrink\cr
|\hfilneg| & cancel |\hfil|\cr
|\vfilneg| & cancel |\vfil|\cr
}
\vskip1em
\item\bull |\hfil| examples
\vskip\baselineskip

\def\borderedbox#1{
\vbox{\hrule
\hbox{\vrule\kern1em\vbox{
\parindent=0pt
\vskip1em #1\vskip1em
}
\kern1em\vrule}
\hrule}}

%% [2025-05-28T09:33:22+08:00] it cannot use borderedbox macro below
%% due to the verbatim content wrapped by `|' would be expanded
%% automatically in a macro and there is no valid approach to mitigate
%% this issue.

\vbox{\hrule
\hbox{\vrule\kern1em\vbox{
\parindent=0pt
\vskip1em
\line{|\line{This text will be flush left\hfil}|}
\line{This text will be flush left.\hfil}
\vskip1em
}
\kern1em\vrule}
\hrule}
\vskip-1pt
\vbox{
\hbox{\vrule\kern1em\vbox{
\parindent=0pt
\vskip1em
\line{|\line{\hfil This text will be flush right.}|}
\line{\hfil This text will be flush right.}
\vskip1em
}
\kern1em\vrule}
\hrule}
\vskip-1pt
\vbox{
\hbox{\vrule\kern1em\vbox{
\parindent=0pt
\vskip1em
\line{|\line{\hfil This text will be centered.\hfil}|}
\line{\hfil This text will be centered.\hfil}
\vskip1em
}
\kern1em\vrule}
\hrule}
\vskip-1pt
\vbox{\hrule
\hbox{\vrule\kern1em\vbox{
\parindent=0pt
\vskip1em
\line{|\line{Some text flush left\hfil and some flush right.}|}
\line{Some text flush left\hfil and some flush right.}
\vskip1em
}
\kern1em\vrule}
\hrule}
\vskip-1pt
\vbox{
\hbox{\vrule\kern1em\vbox{
\parindent=0pt
\vskip1em
\line{|\line{Alpha\hfil centered between Alpha and Omega\hfil Omega}|}
\line{Alpha\hfil centered between Alpha and Omega\hfil Omega}
\vskip1em
}
\kern1em\vrule}
\hrule}
\vskip-1pt
\vbox{
\hbox{\vrule\kern1em\vbox{
\parindent=0pt
\vskip1em
\line{|\line{Five\hfil words\hfil equally\hfil spaced\hfil out.}|}
\line{Five\hfil words\hfil equally\hfil spaced\hfil out.}
\vskip1em
}
\kern1em\vrule}
\hrule}

\vskip\baselineskip

\myhr

\ansno12.2:
Official answer: `` |\line{\hfil\hfil What happens now?\hfil}| is
placed in a line of width\break |\hsize|, with twice as much space at the
left as at the right.''
\vskip.5em
\borderedbox{
\line{\hfil\hfil What happens now?\hfil}
}

`` |\line{\hfill\hfil and now?\hfil}|\ is put flush right on the
following line.''
\vskip.5em
\borderedbox{
\line{\hfill\hfil and now?\hfil}
}

\ansno12.3:
Official answer: ``The first two give an ``overfull box'' if the
argument doesn't fit on a line; the third allows the argument to stick
out into the margins instead. \ (Plain \TeX's ^|\centerline| is
|\centerlinec|; the stickout effect shows up in the narrow-column
experiment of Chapter~6.) \ If the argument contains no infinite glue,
|\centerlinea| and |\centerlineb| produce the same effect; but
|\centerlineb| will center an argument that contains `fil' glue.''

\myhr
\ansno12.4:

\noindent``|Mr.~\& Mrs.~User were married by Rev.~Drofnats,|\break
|who preached on Matt.~19\thinspace:\thinspace3--9.|'' $\rightarrow$\hfil\break
``Mr.~\& Mrs.~User were married by Rev.~Drofnats, who preached on
Matt.~19\thinspace:\thinspace3--9.''

\ansno12.5:
|Donald~E.\ Knuth, ``Mathematical typography,'' {\sl Bull.\
Amer.\ Math.\ Soc.\ \bf1} (1979), 337--372.|
``Donald~E.\ Knuth, ``Mathematical typography,'' {\sl Bull.\
Amer.\ Math.\ Soc.\ \bf1} (1979), 337--372.''

\ansno12.6:
Official answer: ``There are several ways; perhaps the easiest are to
type `|\hbox{NASA}.|'\ or `|NASA\null.|' \ (The ^|\null| macro is an
abbreviation for `|\hbox{}|'.)''

%% [2025-05-30T21:00:50+08:00] Tired of being perceiving from
%% AI. Wasting life ...

\ansno12.7:
Official answer: ``1000, except: 999 after |O|, |B|, |S|, |D|, and
|J|; 1250 after the comma; 3000 after the exclamation point, the
right-quote marks, and the periods. If a period had come just after
the |B| (i.e., if the text had said `|B. Sally|'), the space factor
after that period would have been~1000, not~3000.''

\ansno12.8:
Official answer: ``|\box3| is $2\pt$ high, $4\pt$ deep, $3\pt$ wide.
Starting at the reference point of\/ |\box3|, go right $.75\pt$ and
down $3\pt$ to reach the reference point of\/ |\box1|; or go right
$1\pt$ to reach the reference point of\/ |\box2|.''

\ansno12.9:
Official answer: ``The stretch and shrink components of\/
|\baselineskip| and |\lineskip| should be equal, and the
|\lineskiplimit| should equal the normal |\lineskip| spacing, to
guarantee continuity.''

\ansno12.10:
Official answer: ``Yes it did, but only because none of his boxes had
a negative height or depth. He would have been safer if he had set
|\baselineskip=-1000pt|, |\lineskip=0pt|, and
|\lineskiplimit=16383pt|. \ (Plain \TeX's ^|\offinterlineskip| macro
does this.)''

\ansno12.11:
Official answer: ``The interline glue will be zero, and the natural
height is $1+1-3+2=1\pt$ (because the depth of\/ |\box2| isn't
included in the natural height); so the glue will ultimately become
|\vskip-1pt| when it's set.  Thus, |\box3| is $3\pt$ high, $2\pt$
deep, $4\pt$ wide. Its reference point coincides with that of\/
|\box2|; to get to the reference point of\/ |\box1| you go up $2\pt$
and right $3\pt$.''

\ansno12.12:
Official answer: ``The interline glue will be $6\pt$ minus $3\,{\rm fil}$; the final
depth will be zero, since |\box2| is followed by glue; the natural
height is $12\pt$; and the shrinkability is $5\,{\rm fil}$. So |\box4|
will be $4\pt$ high, $0\pt$ deep, $1\pt$ wide, and it will contain
five items: |\vskip|\penalty0\hbox{|-1.6pt|}, |\box1|, |\vskip1.2pt|,
|\moveleft4pt\box2|, |\vskip-1.6pt|. Starting at the reference point of
|\box4|, you get to the reference point of\/ |\box1| by going up $4.6\pt$,
or to the reference point of\/ |\box2| by going up $.4\pt$ and left $4\pt$.
\ (For example, you go up $4\pt$ to get to the upper left corner of
|\box4|; then down $-1.6\pt$, i.e., up $1.6\pt$, to get to the upper
left corner of\/ |\box1|; then down $1\pt$ to reach its reference
point.  This problem is clearly academic, since it's rather ridiculous
to include infinite shrinkability in the baselineskip.)''

\ansno12.13:
Official answer: ``Now |\box4| will be $4\pt$ high, $-4\pt$ deep,
$1\pt$ wide, and it will contain |\vskip|\penalty0\hbox{|-2.4pt|},
|\box1|, |\vskip-1.2pt|, |\moveleft4pt\box2|, |\vskip-2.4pt|. From the
baseline of\/ |\box4|, go up exactly $5.4\pt$ to reach the baseline
of\/ |\box1|, or exactly $3.6\pt$ to reach the baseline of\/
|\box2|.''

\ansno12.14:
Official answer: ``|\vbox to| $x$|{}| produces height $x$;
|\vtop to| $x$|{}| produces depth $x$; the other dimensions are zero.
\ (This holds even when $x$ is negative.)''

\ansno12.15:
Official answer: ``
There are several possibilities:
\begintt
\def\nullbox#1#2#3{\vbox to#1{\vss\hrule height-#2depth#2width#3}}
\endtt
works because the rule will be of zero thickness. Less tricky is
\begintt
\def\nullbox#1#2#3{\vbox to#1{\vss\vtop to#2{\vss\hbox to#3{}}}}
\endtt
Both of these are valid with negative height and/or depth, but they do
not produce negative width. If the width might be negative, but not the
height or depth, you can use, e.g.,
|\def\nullbox#1#2#3{\hbox to#3{\hss\raise#1\null\lower#2\null}}|.
It's impossible for |\hbox| to construct a box
whose height or depth is negative; it's impossible for |\vbox| or
|\vtop| to construct a box whose width is negative.\par
However, there's actually a trivial solution to the general problem, based on
features that will be discussed later:
\begintt
\def\nullbox#1#2#3{\setbox0=\null
  \ht0=#1 \dp0=#2 \wd0=#3 \box0 }
\endtt
''

\ansno12.16:
Official answer: ``|\def\llap#1{\hbox to 0pt{\hss#1}}|''

\ansno12.17:
Official answer: ``You get `A' at the extreme left and `puzzle.\null'
at the extreme right, because the space between words has the only
stretchability that is finite; the infinite stretchability cancels
out. \ (In this case, \TeX's rule about ^{infinite glue} differs from
what you would get in the limit if the value of $1\,{\rm fil}$ were
finite but getting larger and larger.  The true limiting behavior
would stretch the text `A~puzzle.\null' in the same way, but it would
also move that text infinitely far away beyond the right edge of the
page.)''

\myhr
\vfill
\eject


\noindent Chater 13.
\myhr
\noindent {\bf Summary}:
\halign{\indent\hfil#\ &\ #\hfil\cr
Vertial Mode               & Main Vertical List\cr
Internal Vertial Mode      & Vertical list for vbox\cr
Horizontal Mode            & horizontal list for a paragraph\cr
Restricted Horizontal Mode & horizontal list for an hbox\cr
Math Mode                  & math formula in horizontal list\cr
Display Math Mode          & math formula on a line by itself\cr
}

\myhr

\ansno13.1:
|\hbox| will not work

\ansno13.2:
Official answer: ``The output of\/ |\tracingcommands| shows that four
blank space tokens were digested; these originated at the ends of
lines 2,~3, 4, and~5. Only the first had any effect, since blank
spaces are ignored in math formulas and in vertical modes.''

\ansno13.3:
Official answer: ``The |end-group character| finishes the paragraph and the |\vbox|,
and |\bye| stands for `|\par\vfill...|', so the next three commands are
\begintt
{math mode: math shift character $}
{restricted horizontal mode: end-group character }}
{vertical mode: \par}
\endtt
''

\ansno13.4:
Official answer: ``It contains only mixtures of vertical glue and
horizontal rules whose reference points appear at the left of the
page; there's no text.''

\ansno13.5:
Official answer: ``Vertical mode can occur only as the outermost mode;
horizontal mode and display math mode can occur only when immediately
enclosed by vertical or internal vertical mode; ordinary math mode
cannot be immediately enclosed by vertical or internal vertical mode;
all other cases are possible.''

\myhr
\vfill
\eject


\noindent Chapter 14.
\myhr

\ansno14.1:
\halign{#\hfil\ &\ #\hfil\cr
|(cf.~Chapter~12).|                                &(cf.~Chapter~12).\cr
|Chapters 12 and~21.|                              &Chapters 12 and~21.\cr
|line~16 of Chapter~6's {\tt story}|               &line~16 of Chapter~6's {\tt story}\cr
|lines 7 to~11|                                    &lines 7 to~11\cr
|lines 2,~3, 4, and~5.|                            &lines 2,~3, 4, and~5.\cr
|(2)~a big black bar|                              &(2)~a big black bar\cr
|All 256~characters are initially of category~12,| &All 256~characters are initially of category~12,\cr
|letter~{\tt x} in family~1.|                      &letter~{\tt x} in family~1.\cr
|the factor~$f$, where $n$~is 1000~times~$f$.|     &the factor~$f$, where $n$~is 1000~times~$f$.\cr
}%

\ansno14.2:
`|for all $n$~greater than~$n_0$|' $\rightarrow$ `for all $n$~greater than~$n_0$'

\ansno14.3:
Official answer: `` `|exercise \hbox{4.3.2--15}|' guarantees that
there is no break after the ^{en-dash}. But this precaution is rarely
necessary, so `|exercise 4.3.2--15|' is an acceptable answer. No |~|
is needed; `\hbox{4.3.2--15}' is so long that it causes no offense at
the beginning of a line.''

\ansno14.4:
Official answer: ``The space you get from |~| will stretch or shrink
with the other spaces in the same line, but the space inside an hbox
has a fixed width since that glue has already been set once and for
all.  Furthermore the first alternative permits the word Chapter to be
^{hyphenate}d.''

\ansno14.5:
Official answer: ```|\hbox{$x=0$}|' is unbreakable, and we will see
later that `|${x=0}$|' cannot be broken. Both of these solutions set
the glue surrounding the equals sign to some fixed value, but such
glue normally wants to stretch; furthermore, the |\hbox| solution
might include undesirable blank space at the beginning or end of a
line, if\/ ^|\mathsurround| is nonzero. A third solution
`|$x=\nobreak0$|' avoids both defects.''

\ansno14.6:
Official answer: ``|\exhyphenpenalty=10000| prohibits all such breaks,
according to the rules found later in this chapter. Similarly,
|\hyphenpenalty=10000| prevents breaks after implicit (discretionary)
hyphens.''

\myhr

\noindent{\bf Summary}
{\obeylines\smallskip
Roses are red,
\quad Violets are blue;
Rhymes can be typeset
\quad With boxes and glue.
\smallskip}

\myhr

\ansno14.7:
Official answer: ``The second and fourth lines are indented by an additional ``quad''
of space, i.e., by one extra em in the current type style.
\ (The control sequence |\quad| does an ^|\hskip|; when \TeX\ is in
vertical mode, |\hskip| begins a new paragraph and puts glue after the
indentation.) \ If\/ |\indent| had been used instead, those lines
wouldn't have been indented any more than the first and third, because
|\indent| is implicit at the beginning of every paragraph.  Double
indentation on the second and fourth lines could have been achieved by
`|\indent\indent|'.''

\myhr

\noindent{\bf Summary}
\vskip1em
\noindent\item\bull Each item in a horizontal list is one of the following types of things

\halign{\hfil#\ &\ #\hfil\cr
a box                 & (a character or ligature or rule or hbox or vbox)\cr
a discretionary break & (to be explained momentarily)\cr
a ``whatsit''         & (something special to be explained later)\cr
vertical material     & (from |\mark| or |\vadjust| or |\insert|)\cr
a glob of glue        & (or |\leaders|, as we will see later)\cr
a kern                & (something like glue that doesn’t stretch or shrink)\cr
a penalty             & (representing the undesirability of breaking here)\cr
``math-on''           & (beginning a formula) or “math-off” (ending a formula)\cr
}
\vskip1em

\noindent\item\bull4 types {\it discardable}:
\halign{\hfil#\ &\ #\hfil\cr
non-discardable & box \cr
non-discardable & discretionary break \cr
non-discardable & ``whatsit'' \cr
non-discardable & vertical material \cr
discardable & glue\cr
discardable & kern\cr
discardable & penalty\cr
discardable & math items\cr
}

\myhr


%% [2025-05-22T07:09:46+08:00] display \vsize and \hsize
%% \message{The current \string\vsize\space is \the\vsize.}
%% \message{The current \string\hsize\space is \the\hsize.}
\bye
