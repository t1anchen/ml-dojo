%% \tracingall
\input manmac

\noindent Chapter 1
\vskip.5\baselineskip\hrule\vskip.5\baselineskip
\ansno1.1:
 A \TeX nician (underpaid); sometimes also called a \TeX acker.
\vskip.5\baselineskip\hrule\vskip.5\baselineskip
\vfill
\eject


\noindent Chapter 2

\vskip.5\baselineskip\hrule\vskip.5\baselineskip
%% [2025-05-06T18:14:46+08:00]
``I understand''

%% [2025-05-06T18:12:00+08:00] Do not write comment inside the block
%% between `\begintt' and `\endtt' because the anything between
%% `\begintt' and `\endtt' will be eventually printed out in output
%% pdf.

%% [2025-05-06T18:16:22+08:00] `\]' represents "visible spacebar" in
%% normal/display mode
Visible spacebar: ``I\]understand.''

%% [2025-05-06T18:13:44+08:00] `|]' represents "visible spacebar" in
%% verbatim mode
\begintt
Verbatim spacebar: ``I|]understand.''
\endtt

\begindisplay
|-| & hyphen (-)\cr
|--| & en-dash (--)\cr
|---| & em-dash (---)\cr
|$-$| & minus sign $-$\cr
\enddisplay
\hrule

\ansno2.1: Alice said, ``I always use an en-dash instead of a hyphen
when specifying page numbers like `480--491' in a bibliography.''


\ansno2.2: Four hyphens `|----|' becomes one em-dash and one hyphen
`----'

\ansno2.3: fluff

\vskip.5\baselineskip\hrule\vskip.5\baselineskip

{\bf Summary}:
\item\bull ligatures
\item\bull kerning
\item\bull alternative input methods for left quotes, right quotes

\begindisplay
{\tt |\lq\lq\]I\]understand.\rq\rq|} $\Rightarrow$ \lq\lq\]I\]understand.\rq\rq
\enddisplay
\hrule


\ansno2.4: ``\thinspace` and `\thinspace``


\ansno2.5: Following Occam's Razor Principle, introducing extra
complexity on typewriting annotations might not be a wise move because
it increases extra maintenance cost in minds and memory. Besides,
introducing dollar sign might cause confusions between {\sl Math mode}
and {\sl Text mode} that it may deteriorate maintenability of code.

\vskip.5\baselineskip\hrule\vskip.5\baselineskip

\vfill
\eject


\noindent Chapter 3.
\vskip.5\baselineskip\hrule\vskip.5\baselineskip

Note: |ESC| key $\neq$ escape character

{\bf Summary}: control sequence, control symbol

\vskip.5\baselineskip\hrule\vskip.5\baselineskip
\ansno3.1:
`|\I'm \exercise3.1\\!|' control sequences are `|\I|', `|\exercise|',
 and `|\\|'

\ansno3.2:
\begindisplay
`math\'ematique' & $\rightarrow$\quad |`math\'ematique'|\cr
`centim\`etre' & $\rightarrow$\quad |`centim\`etre'|\cr
\enddisplay

\vskip.5\baselineskip\hrule\vskip.5\baselineskip

note: \TeX\ ignores spaces after control worlds

summary: apprimately 300 of 900 control sequence in \TeX\ are {\sl
primitive}

\vskip.5\baselineskip\hrule\vskip.5\baselineskip

\ansno3.3: |\|\] is primitive, while |\|\<return> is represented as |^^M|

\ansno3.4:
([2025-05-09T00:11:26+08:00] after asking Perplexity.AI about how to
understand Knuth's answer, it makes sense now)

For length~2 case, the first position is always occupied by `|\|', the
remaining position is for a single ASCII character, which is 256
different variants (in the problem description, it said `including
escape character').

For length~3, it might be strictly limited to alphabet characters
(i.e. |[A-Za-z]| if represented in Regex). The first position is
contantly as `|\|', and each of the remaining positions has 52
variants, which is $52^2$ possible control sequences.
\vskip.5\baselineskip\hrule\vskip.5\baselineskip

\vfill
\eject


\noindent Chapter 4.
\vskip.5\baselineskip\hrule\vskip.5\baselineskip
\begindisplay
|\rm| & {\rm Roman}\cr
|\sl| & {\sl Slanted}\cr
|\it| & {\it Italic}\cr
|\tt| & {\tt Typewriter}\cr
|\bf| & {\bf Bold}\cr
\enddisplay
\vskip.5\baselineskip\hrule\vskip.5\baselineskip

\ansno4.1:
`Ulrich Dieter, {\sl Journal f\"ur die reine und angewandte
Mathematik} {\bf 201} (1959), 37--70.'

\ansno4.2:
{\it Explain how to typeset a\/ {\rm roman} word in the midst of an
italicized sentence.}

\ansno4.3:
|\def\ic#1{\setbox0=\hbox{#1\/}\dimen0=\wd0|\parbreak
|\setbox0=\hbox{#1}\advance\dimen0 by -\wd0 }|.

\vskip.5\baselineskip\hrule\vskip.5\baselineskip

\tenrm smaller \ninerm and smaller
\eightrm and smaller \sevenrm and smaller
\sixrm and smaller \fiverm and smaller \tenrm

\vskip.5\baselineskip\hrule\vskip.5\baselineskip

\ansno4.4:
The numeric ones would be possibly interpreted as `|\1|' and `0point'
instead of the entire control sequence.

\ansno4.5:
Quoted from official answer: ``Say |\def\sl{\it}| at the beginning,
and delete other definitions of\/ |\sl| that might be present in your
format file (e.g., there might be one inside a |\tenpoint| macro).''

%% \vskip.5\baselineskip\hrule\vskip.5\baselineskip

%% [2025-05-12T09:43:23+08:00] If it's ttf/otf font, it is required to
%% be preprocessed by mktexmf first, otherwise an error will be thrown
%% out
%%
%%     kpathsea: Running mktextfm Cambria
%%
%%     The command name is $TEXLIVE_HOME\2024\bin\windows\mktextfm
%%
%%     kpathsea: Running mktexmf Cambria.mf
%%
%%     The command name is $TEXLIVE_HOME\2024\bin\windows\mktexmf
%%     name = Cambria, rootname = Cambria, pointsize =
%%     mktexmf: empty or non-existent rootfile!
%%     Cannot find Cambria.mf.
%%     kpathsea: Appending font creation commands to build/missfont.log.
%%
%% ----
%% \font\ninerm=Cambria at 10pt

%% \vskip.5\baselineskip\hrule\vskip.5\baselineskip

\ansno4.6:
|\font\tenrm=cmr10 at 5pt|\parbreak
|\font\tenrm=cmr10 scaled 500|

Note: In the official answer, it uses `|\squinttenrm|'

\vskip.5\baselineskip\hrule\vskip.5\baselineskip

\font\msb=msbm10 at 10pt
\def\ldots{\mathinner{\ldotp\ldotp\ldotp}}

{\bf Summary}:
\item\bull magnification ratio factor
\item\bull $1.2^n, n \in $\thinspace {\msb N}
\item\bull |\magstep0|, |\magstep1|, $\ldots$, |\magstep5|

\vskip.5\baselineskip\hrule\vskip.5\baselineskip

\vfill
\eject


\noindent Chapter 5.

\vskip.5\baselineskip\hrule\vskip.5\baselineskip

{\bf Summary}: Use `|{|', `|}|' for grouping

\vskip.5\baselineskip\hrule\vskip.5\baselineskip

\ansno5.1:
Use grouping: `|shelf{f}ul|'\ $\rightarrow$\ `shelf{f}ul' rather than `shelfful'\parbreak

Official answer: |{shelf}ful| or |shelf{}ful|, etc.; or even
|shelf\/ful|, which yields a shelf\/ful instead of a
shelf{\kern0pt}ful.

\ansno5.2:
Use grouping: `|{\]}{\]}{\]}|' $\rightarrow$ `{\]}{\]}{\]}'\parbreak

Official answer: `\]|{|\]|}|\]' or `\]|{}|\]|{}|\]', etc. Plain \TeX\
also has a |\space| macro, so you can type |\space\space\space|.  \
(These aren't strictly equivalent to `|\|\]|\|\]|\|\]', since they
adjust the spaces by the current ``{space factor},'' as explained
later.)

\vskip.5\baselineskip\hrule\vskip.5\baselineskip

{\bf Summary}:
\item\bull |\centerline|
\item\bull grouping
\item\bull scope
\item\bull nested

Example:
\begindisplay
\centerline{This information should be {\it centered}.}
\enddisplay

\vskip.5\baselineskip\hrule\vskip.5\baselineskip

\ansno5.3:
First case: same\parbreak
second case: Only `S' centred but rest would be automatically arranged.

\ansno5.4:
The line will be centred and `centered' will be italic.

\ansno5.5:
\def\ital#1{{\it #1\/}}\parbreak
|\def\ital#1{{\it #1\/}}|

|\ital{hello} world| $\rightarrow$ \ital{hello} world

\item\bull Pros: scoped, more legible
\item\bull Cons: learning cost

%% \global\advance\count0 by 1

\ansno5.6:
Official answer: |{1 {2 3 4 5} 4 6} 4|

%% \def\c#1{\count1=#1}
%% \def\g{\global\count1=}
%% \def\s{\showthe\count1}

%% \hbox{%
%%   \c1\s
%%   \g2
%%   {%
%%     \s
%%     \c3\s
%%     \g4\s
%%     \c5\s
%%   }%
%%   \s
%%   \c6\s
%% }%
%% \s

%% ===================================================================
%% [2025-05-13T19:40:35+08:00] Cannot reproduce the result neither in
%% TeX nor in pdfTeX. It keeps alarming errors like
%%
%%     <to be read again>
%%                        \global
%%     \g ->\global
%%                  \count 1=
%%     l.268   \g
%%               2

%% it does not work even braces added/removed or marco parameter
%% added/removed. It seems like the behaviorial differences between
%% different versions of TeX.

\vskip.5\baselineskip\hrule\vskip.5\baselineskip

\vfill
\eject

\noindent Chapter 6.

\vskip.5\baselineskip\hrule\vskip.5\baselineskip

$$\halign{\hbox to\parindent{\hfil\sevenrm#\ \ }&#\hfil\cr
1&|\hrule|\cr
2&|\vskip 1in|\cr
3&|\centerline{\bf A SHORT STORY}|\cr
4&|\vskip 6pt|\cr
5&|\centerline{\sl by A. U. Thor}|\cr
6&|\vskip .5cm|\cr
7&|Once upon a time, in a distant|\cr
8&|  galaxy called \"O\"o\c c,|\cr
9&|there lived a computer|\cr
10&|named R.~J. Drofnats.|\cr
11&||\cr
12&|Mr.~Drofnats---or ``R. J.,'' as|\cr
13&|he preferred to be called---|\cr
14&|was happiest when he was at work|\cr
15&|typesetting beautiful documents.|\cr
16&|\vskip 1in|\cr
17&|\hrule|\cr
18&|\vfill\eject|\cr}$$

\vskip.5\baselineskip

\hrule
\vskip 1in
\centerline{\bf A SHORT STORY}
\vskip 6pt
\centerline{\sl by A. U. Thor}
\vskip .5cm
Once upon a time, in a distant
  galaxy called \"O\"o\c c,
there lived a computer
named R.~J. Drofnats.

Mr.~Drofnats---or ``R. J.,'' as
he preferred to be called---
was happiest when he was at work
typesetting beautiful documents.
\vskip 1in
\hrule
\vfill\eject

\vskip.5\baselineskip\hrule\vskip.5\baselineskip

{\bf Summary}:

\item\bull `|\c|' (cedilla)
\item\bull `|~|' (ties, treat as normal space but not
to break between lines, e.g. names)

\vskip.5\baselineskip\hrule\vskip.5\baselineskip

\ansno6.1:
Official answer: ``Laziness and/or obstinacy''

\ansno6.2:
Official answer: `` There's an unwanted space after `called---',
because (as the book says) \TeX\ treats the end of a line as if it
were a blank space. That blank space is usually what you want, except
when a line ends with a hyphen or a dash; so you should WATCH OUT for
lines that end with hyphens or dashes''

\ansno6.3:
Official answer: ``It represents the heavy bar that shows up in your
output. \ (This bar wouldn't be present if had been set to |0pt|, nor
is it present in an underfull box.)''

\ansno6.4:
Official answer: ``This is the |\parfillskip| space that ends the
paragraph.  In plain \TeX\ the parfillskip is zero when the last line
of the paragraph is full; hence no space actually appears before the
rule in the output of Experiment~3. But all hskips show up as spaces
in an overfull box message, even if they're zero.''

\ansno6.5:
Official answer: ``Run \TeX\ with \hbox{|\hsize=1.5in|} \hbox{|\tolerance=10000|}
\hbox{|\raggedright|} \hbox{|\hbadness=-1|} and then |\input story|. \TeX\ will
report the badness of all lines (except the final lines of paragraphs,
where fill glue makes the badness zero).''

\ansno6.6:
Official answer: ``|\def\extraspace{\nobreak \hskip 0pt plus
.15em\relax}|\parbreak
|\def\dash{\unskip\extraspace---\extraspace}|\par\nobreak\smallskip\noindent
(If you try this with the story at 2-inch and 1.5-inch sizes, you will
notice a substantial improvement. The |\unskip| allows people to leave
a space before typing |\dash|.  \TeX\ will try to hyphenate before
|\dash|, but not before `|---|'; cf.\ Appendix~H\null. The ^|\relax|
at the end of |\extraspace| is a precaution in case the next word is
`|minus|'.)''

\ansno6.7:
Official answer: ``\TeX\ would have deleted five tokens: |1|, |i|,
|n|, \], |\centerline|.  (The space was at the end of line~2, the
|\centerline| at the beginning of line~3.)''

\ansno6.8:
Official answer: ``A control sequence like |\centerline| might well
define a control sequence like |\ERROR| before telling \TeX\ to look
at |#1|. Therefore
\TeX\ doesn't interpret control sequences when it scans an argument.''

\vskip.5\baselineskip\hrule\vskip.5\baselineskip
\vfill
\eject


\noindent Chapter 7.
\vskip.5\baselineskip\hrule\vskip.5\baselineskip

{\bf Summary}:
\item\bull Reserved $10$ characters (must be escaped with `|\|'): `|\|', `|{|', `|}|', `|$|', `|&|', `|#|', `|^|', `|_|', `|%|', `|~|'
\item\bull example: `|$\{a \backslash b\}$|' $\rightarrow$ `$\{a \backslash b\}$'

\vskip.5\baselineskip\hrule\vskip.5\baselineskip

\ansno7.1:
The `|&|', `|$|' and `|%|' must be escaped.

\ansno7.2:
Official answer: ``Reverse slashes (backslashes) are fairly uncommon
in formulas or text, and |\\| is very easy to type; it was therefore
felt best not to reserve |\\| for such limited use. Typists can define
|\\| to be whatever they want (including |\backslash|).''

\vskip.5\baselineskip\hrule\vskip.5\baselineskip

{\bf Summary:} \cstok{boxed token}

\vskip.5\baselineskip\hrule\vskip.5\baselineskip

\ansno7.3:
Official answer: ``1, 2, 3, 4, 6, 7, 8, 10, 11, 12, 13. {Active
characters} (type 13) are somewhat special; they behave like control
sequences in most cases (e.g., when you say `|\let||\x=~|' or
`|\ifx||\x~|'), but they behave like character tokens when they appear
in the token list of\/ |\uppercase| or |\lowercase|, and when
unexpanded after |\if| or |\ifcat|.''

\ansno7.4:
Official answer: ``It ends with either |>| or |}| or any character of
category 2; then the effects of all |\catcode| definitions within the
group are wiped out, except those that were |\global|.''

\ansno7.5:
Official answer: ``If you type `|\message{\string~}|' and `|\message{\string\~}|', \TeX\
responds with `|~|' and `|\~|', respectively. ^^|\message|
To get |\|$_{12}$ from |\string| you therefore need to make backslash an
active character. One way to do this is
\begintt
{\catcode`/=0 \catcode`\\=13 /message{/string\}}
\endtt
(The ``^{null control sequence}'' that you get when there are no
tokens between |\csname| and |\endcsname| is not a solution to this exercise,
because |\string| converts it to `|\csname\endcsname|'. There is, however,
another solution: If \TeX's |\escapechar| parameter---which will be
explained in one of the next dangerous bends---is negative or greater
than~255, then `|\string\\|' works.)''

\ansno7.6:
|\|$_{12}$ |a|$_{12}$ |\|$_{12}$ \]$_{10}$ |b|$_{12}$.

\ansno7.7:
Official answer: ``|\def\ifundefined#1{\expandafter\ifx\csname#1\endcsname\relax}|%
\hfil\break Note that a control sequence like this must be used with care;
it cannot be included in {conditional} text, because the |\ifx| will not
be seen when |\ifundefined| isn't expanded.''

\ansno7.8:
|\uppercase{a\lowercase{bC}}| $\rightarrow$ `\uppercase{a\lowercase{bC}}'

\ansno7.9:
`\copyright\ \uppercase\expandafter{\romannumeral\year}' $\rightarrow$ `|\copyright\ \uppercase\expandafter{\romannumeral\year}|'

\ansno7.10:
Official answer:
\begintt
\def\gobble#1{} % remove one token
\def\appendroman#1#2#3{\expandafter\def\expandafter#1\expandafter
  {\csname\expandafter\gobble\string#2\romannumeral#3\endcsname}}
\endtt

\vskip.5\baselineskip\hrule\vskip.5\baselineskip

\vfill
\eject


\noindent Chapter 8.

\vskip.5\baselineskip\hrule\vskip.5\baselineskip

{\bf Summary}
`|\char98 u\char98\char98 le|' $\rightarrow$ `\char98 u\char98\char98 le'
\halign{# && $\rightarrow$ #\cr
`|\char98|' & `\char98'\cr
`|\char'142|' & `\char'142'\cr
`|\char"62|' & `\char"62'\cr
}


\vskip.5\baselineskip\hrule\vskip.5\baselineskip

\ansno8.1:

|%| in \TeX can be interpreted as reserved keyword of comment
|%| character.

Official answer: ``The |%| would be treated as a comment character,
because its category code is~14; thus, no |%| token or |}| token would
get through to the gullet of \TeX\ where numbers are treated. When a
character is of category 0, 5, 9, 14, or~15, the extra |\| must be
used; and the |\| doesn't hurt, so you can always use it to be safe.''

\ansno8.2:
\halign{#\ &\ #\hfil\cr
(a)&A caracter of category~5 can be converted into `\]'$_{10}$ or
a \cstok{par} token, \cr
&while a character of category~14 never produces a token.\cr
(b)&Different category numbers\cr
(c)&Same as (b)\cr
(d)&No\cr
(e)&Yes\cr
(f)&No\cr
}

\vskip.5\baselineskip\hrule\vskip.5\baselineskip

\ansno8.3: TBD
\ansno8.4: TBD
\ansno8.5: TBD
\ansno8.6: TBD
\ansno8.7: TBD

\vskip.5\baselineskip\hrule\vskip.5\baselineskip
\vfill
\eject


\noindent Chapter 9.
\vskip.5\baselineskip\hrule\vskip.5\baselineskip

Summary:

upper accent

$$\halign{\indent\hbox to 50pt{#\hfil}&\hbox to 35pt{#\hfil}&#\hfil\cr
\it\negthinspace Type&\it to get\cr
\noalign{\smallskip}
|\`o|&\`o&(grave accent)\cr
|\'o|&\'o&(acute accent)\cr
|\^o|&\^o&(circumflex or ``hat'')\cr
|\"o|&\"o&(umlaut or dieresis)\cr
|\~o|&\~o&(tilde or ``squiggle'')\cr
|\=o|&\=o&(macron or ``bar'')\cr
|\.o|&\.o&(dot accent)\cr
|\u o|&\u o&(breve accent)\cr
|\v o|&\v o&(h\'a\v cek or ``check'')\cr
|\H o|&\H o&(long Hungarian umlaut)\cr
|\t oo|&\t oo&(tie-after accent)\cr}$$

accents that go underneath

$$\halign{\indent\hbox to 50pt{#\hfil}&\hbox to 35pt{#\hfil}&#\hfil\cr
\it\negthinspace Type&\it to get\cr
\noalign{\smallskip}
|\c o|&\c o&(cedilla accent)\cr
|\d o|&\d o&(dot-under accent)\cr
|\b o|&\b o&(bar-under accent)\cr}$$

Special letters

$$\halign{\indent\hbox to 50pt{#\hfil}&\hbox to 35pt{#\hfil}&#\hfil\cr
\it\negthinspace Type&\it to get\cr
\noalign{\smallskip}
|\oe,\OE|&\oe,\thinspace\OE&(French ligature OE)\cr
|\ae,\AE|&\ae,\thinspace\AE&(Latin ligature and Scandinavian letter AE)\cr
|\aa,\AA|&\aa,\thinspace\AA&(Scandinavian A-with-circle)\cr
|\o,\O|&\o,\thinspace\O&(Scandinavian O-with-slash)\cr
|\l,\L|&\l,\thinspace\L&(Polish suppressed-L)\cr
|\ss|&\ss&(German ``es-zet'' or sharp S)\cr}$$

\vskip.5\baselineskip\hrule\vskip.5\baselineskip

\ansno9.1: `|na\"\i ve|' $\rightarrow$ `na\"\i ve'

\ansno9.2:
Official answer: ``Belov\`ed prot\'eg\'e; r\^ole co\"ordinator;
souffl\'es, cr\^epes, p\^at\'es, etc.''

\ansno9.3:
`|\AE sop's \OE uvres en fran\c cais|' $\rightarrow$ `\AE sop's \OE
uvres en fran\c cais'

\ansno9.4:
`|{\sl Commentarii Academi\ae\ scientiarum imperialis|\hfil\break
|petropolitan\ae\/} became {\sl Akademi\t\i a Nauk SSSR,| \hfil\break
|Doklady}|' $\rightarrow$ `{\sl Commentarii Academi\ae\ scientiarum
imperialis petropolitan\ae\/} became {\sl Akademi\t\i a Nauk SSSR,
Doklady}'

\ansno9.5:
\halign{#$\rightarrow$&\ #\hfil\cr
`|Ernesto {Ces\`aro}|'                           & `Ernesto {Ces\`aro}'\cr
`|P\'al {Erd\H os}|'                             & `P\'al {Erd\H os}'\cr
`|\O ystein {Ore}|'                              & `\O ystein {Ore}'\cr
`|Stanis\l aw \'Swierczkowski|'                  & `Stanis\l aw \'Swierczkowski' \cr
`|Serge\u\i\ \t Iur'ev|'                         & `Serge\u\i\ \t Iur'ev'\cr
`|Mu\d hammad ibn M\^us\^a {al-Khw\^arizm\^\i}|' & `Mu\d hammad ibn M\^us\^a {al-Khw\^arizm\^\i}'\cr
}

\ansno9.6:
`|{\tt P\'al Erd{\bf\H{\tt o}}s}|' $\rightarrow$ `{\tt P\'al Erd{\bf\H{\tt o}}s}'

\vskip.5\baselineskip\hrule\vskip.5\baselineskip

$$\halign{\indent#\hfil\ &\hfil#\hfil&#\hfil\cr
\it\negthinspace Type&\it to get\cr
\noalign{\smallskip}
|\dag|&\dag&(dagger or obelisk)\cr
|\ddag|&\ddag&(double dagger or diesis)\cr
|\S|&\S&(section number sign)\cr
|\P|&\P&(paragraph sign or pilcrow)\cr}$$

\vskip.5\baselineskip\hrule\vskip.5\baselineskip

\ansno9.7:
`|{\it Europe on {\sl\$}15.00 a day\/}|' $\rightarrow$ `{\it Europe on
{\sl\$}15.00 a day\/}'

\ansno9.8:
Official answer: ``The extra braces keep font changes local. An
argument makes the use of\/ |\'| more consistent with the use of other
accents like |\d|, which are manufactured from other characters
without using the |\accent| primitive.''

\vskip.5\baselineskip\hrule\vskip.5\baselineskip

\vfill
\eject


\noindent Chapter 10.

\vskip.5\baselineskip\hrule\vskip.5\baselineskip

Units:

$$\halign{\indent\tt#&\quad#\hfil\cr
pt&point (baselines in this manual are $12\pt$ apart)\cr
pc&pica ($\rm1\,pc=12\,pt$)\cr
in&inch ($\rm1\,in=72.27\,pt$)\cr
bp&big point ($\rm72\,bp=1\,in$)\cr
cm&centimeter ($\rm2.54\,cm=1\,in$)\cr
mm&millimeter ($\rm10\,mm=1\,cm$)\cr
dd&didot point ($\rm1157\,dd=1238\,pt$)\cr
cc&cicero ($\rm1\,cc=12\,dd$)\cr
sp&scaled point ($\rm65536\,sp=1\,pt$)\cr}$$

\vskip.5\baselineskip\hrule\vskip.5\baselineskip

\ansno10.1:
254 centimeters $=$ 100 in $=$ 7227 pt

\vskip.5\baselineskip\hrule\vskip.5\baselineskip

\bye
