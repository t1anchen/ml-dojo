%% \tracingall
\input manmac

\noindent Chapter 1
\vskip.5\baselineskip\hrule\vskip.5\baselineskip
\ansno1.1:
 A \TeX nician (underpaid); sometimes also called a \TeX acker.
\vskip.5\baselineskip\hrule\vskip.5\baselineskip
\vfill
\eject


\noindent Chapter 2

\vskip.5\baselineskip\hrule\vskip.5\baselineskip
%% [2025-05-06T18:14:46+08:00]
``I understand''

%% [2025-05-06T18:12:00+08:00] Do not write comment inside the block
%% between `\begintt' and `\endtt' because the anything between
%% `\begintt' and `\endtt' will be eventually printed out in output
%% pdf.

%% [2025-05-06T18:16:22+08:00] `\]' represents "visible spacebar" in
%% normal/display mode
Visible spacebar: ``I\]understand.''

%% [2025-05-06T18:13:44+08:00] `|]' represents "visible spacebar" in
%% verbatim mode
\begintt
Verbatim spacebar: ``I|]understand.''
\endtt

\begindisplay
|-| & hyphen (-)\cr
|--| & en-dash (--)\cr
|---| & em-dash (---)\cr
|$-$| & minus sign $-$\cr
\enddisplay
\hrule

\ansno2.1: Alice said, ``I always use an en-dash instead of a hyphen
when specifying page numbers like `480--491' in a bibliography.''


\ansno2.2: Four hyphens `|----|' becomes one em-dash and one hyphen
`----'

\ansno2.3: fluff

\vskip.5\baselineskip\hrule\vskip.5\baselineskip
Summary: ligatures, kerning


alternative input methods for left quotes, right quotes
\begindisplay
{\tt |\lq\lq\]I\]understand.\rq\rq|} $\Rightarrow$ \lq\lq\]I\]understand.\rq\rq
\enddisplay
\hrule


\ansno2.4: ``\thinspace` and `\thinspace``


\ansno2.5: Following Occam's Razor Principle, introducing extra
complexity on typewriting annotations might not be a wise move because
it increases extra maintenance cost in minds and memory. Besides,
introducing dollar sign might cause confusions between {\sl Math mode}
and {\sl Text mode} that it may deteriorate maintenability of code.

\vskip.5\baselineskip\hrule\vskip.5\baselineskip

\vfill
\eject


\noindent Chapter 3.
\vskip.5\baselineskip\hrule\vskip.5\baselineskip

Note: |ESC| key $\neq$ escape character

Summary: control sequence, control symbol

\vskip.5\baselineskip\hrule\vskip.5\baselineskip
\ansno3.1:
`|\I'm \exercise3.1\\!|' control sequences are `|\I|', `|\exercise|',
 and `|\\|'

\ansno3.2:
\begindisplay
`math\'ematique' & $\rightarrow$\quad |`math\'ematique'|\cr
`centim\`etre' & $\rightarrow$\quad |`centim\`etre'|\cr
\enddisplay

\vskip.5\baselineskip\hrule\vskip.5\baselineskip

note: \TeX\ ignores spaces after control worlds

summary: apprimately 300 of 900 control sequence in \TeX\ are {\sl
primitive}

\vskip.5\baselineskip\hrule\vskip.5\baselineskip

\ansno3.3: |\|\] is primitive, while |\|\<return> is represented as |^^M|

\ansno3.4:
([2025-05-09T00:11:26+08:00] after asking Perplexity.AI about how to
understand Knuth's answer, it makes sense now)

For length~2 case, the first position is always occupied by `|\|', the
remaining position is for a single ASCII character, which is 256
different variants (in the problem description, it said `including
escape character').

For length~3, it might be strictly limited to alphabet characters
(i.e. |[A-Za-z]| if represented in Regex). The first position is
contantly as `|\|', and each of the remaining positions has 52
variants, which is $52^2$ possible control sequences.
\vskip.5\baselineskip\hrule\vskip.5\baselineskip

\vfill
\eject


\noindent Chapter 4.
\vskip.5\baselineskip\hrule\vskip.5\baselineskip
\begindisplay
|\rm| & {\rm Roman}\cr
|\sl| & {\sl Slanted}\cr
|\it| & {\it Italic}\cr
|\tt| & {\tt Typewriter}\cr
|\bf| & {\bf Bold}\cr
\enddisplay
\vskip.5\baselineskip\hrule\vskip.5\baselineskip

\ansno4.1:
`Ulrich Dieter, {\sl Journal f\"ur die reine und angewandte
Mathematik} {\bf 201} (1959), 37--70.'

\ansno4.2:
{\it Explain how to typeset a\/ {\rm roman} word in the midst of an
italicized sentence.}

\ansno4.3:
|\def\ic#1{\setbox0=\hbox{#1\/}\dimen0=\wd0|\parbreak
|\setbox0=\hbox{#1}\advance\dimen0 by -\wd0 }|.

\vskip.5\baselineskip\hrule\vskip.5\baselineskip

\tenrm smaller \ninerm and smaller
\eightrm and smaller \sevenrm and smaller
\sixrm and smaller \fiverm and smaller \tenrm

\vskip.5\baselineskip\hrule\vskip.5\baselineskip

\ansno4.4:
The numeric ones would be possibly interpreted as `|\1|' and `0point'
instead of the entire control sequence.

\ansno4.5:
Quoted from official answer: ``Say |\def\sl{\it}| at the beginning,
and delete other definitions of\/ |\sl| that might be present in your
format file (e.g., there might be one inside a |\tenpoint| macro).''

%% \vskip.5\baselineskip\hrule\vskip.5\baselineskip

%% [2025-05-12T09:43:23+08:00] If it's ttf/otf font, it is required to
%% be preprocessed by mktexmf first, otherwise an error will be thrown
%% out
%%
%%     kpathsea: Running mktextfm Cambria
%%
%%     The command name is $TEXLIVE_HOME\2024\bin\windows\mktextfm
%%
%%     kpathsea: Running mktexmf Cambria.mf
%%
%%     The command name is $TEXLIVE_HOME\2024\bin\windows\mktexmf
%%     name = Cambria, rootname = Cambria, pointsize =
%%     mktexmf: empty or non-existent rootfile!
%%     Cannot find Cambria.mf.
%%     kpathsea: Appending font creation commands to build/missfont.log.
%%
%% ----
%% \font\ninerm=Cambria at 10pt

%% \vskip.5\baselineskip\hrule\vskip.5\baselineskip

\ansno4.6:
|\font\tenrm=cmr10 at 5pt|\parbreak
|\font\tenrm=cmr10 scaled 500|

Note: In the official answer, it uses `|\squinttenrm|'

\vskip.5\baselineskip\hrule\vskip.5\baselineskip

\font\msb=msbm10 at 10pt
\def\ldots{\mathinner{\ldotp\ldotp\ldotp}}

Summary: magnification ratio factor, $1.2^n, n \in $\thinspace {\msb N},
|\magstep0|, |\magstep1|, $\ldots$, |\magstep5|

\vskip.5\baselineskip\hrule\vskip.5\baselineskip

\vfill
\eject


\noindent Chapter 5.

\vskip.5\baselineskip\hrule\vskip.5\baselineskip

Summary: Use `|{|', `|}|' for grouping

\vskip.5\baselineskip\hrule\vskip.5\baselineskip

\ansno5.1:
Use grouping: `|shelf{f}ul|'\ $\rightarrow$\ `shelf{f}ul' rather than `shelfful'\parbreak

Official answer: |{shelf}ful| or |shelf{}ful|, etc.; or even
|shelf\/ful|, which yields a shelf\/ful instead of a
shelf{\kern0pt}ful.

\ansno5.2:
Use grouping: `|{\]}{\]}{\]}|' $\rightarrow$ `{\]}{\]}{\]}'\parbreak

Official answer: `\]|{|\]|}|\]' or `\]|{}|\]|{}|\]', etc. Plain \TeX\
also has a |\space| macro, so you can type |\space\space\space|.  \
(These aren't strictly equivalent to `|\|\]|\|\]|\|\]', since they
adjust the spaces by the current ``{space factor},'' as explained
later.)

\vskip.5\baselineskip\hrule\vskip.5\baselineskip

Summary: |\centerline|, grouping, scope, nested

Example:
\begindisplay
\centerline{This information should be {\it centered}.}
\enddisplay

\vskip.5\baselineskip\hrule\vskip.5\baselineskip

\ansno5.3:
First case: same\parbreak
second case: Only `S' centred but rest would be automatically arranged.

\ansno5.4:
The line will be centred and `centered' will be italic.

\ansno5.5:
\def\ital#1{{\it #1\/}}\parbreak
|\def\ital#1{{\it #1\/}}|

|\ital{hello} world| $\rightarrow$ \ital{hello} world

\item - Pros: scoped, more legible
\item - Cons: learning cost

%% \global\advance\count0 by 1

\ansno5.6:
Official answer: |{1 {2 3 4 5} 4 6} 4|

%% \def\c#1{\count1=#1}
%% \def\g{\global\count1=}
%% \def\s{\showthe\count1}

%% \hbox{%
%%   \c1\s
%%   \g2
%%   {%
%%     \s
%%     \c3\s
%%     \g4\s
%%     \c5\s
%%   }%
%%   \s
%%   \c6\s
%% }%
%% \s

%% ===================================================================
%% [2025-05-13T19:40:35+08:00] Cannot reproduce the result neither in
%% TeX nor in pdfTeX. It keeps alarming errors like
%%
%%     <to be read again>
%%                        \global
%%     \g ->\global
%%                  \count 1=
%%     l.268   \g
%%               2

%% it does not work even braces added/removed or marco parameter
%% added/removed. It seems like the behaviorial differences between
%% different versions of TeX.

\vskip.5\baselineskip\hrule\vskip.5\baselineskip

\vfill
\eject

\noindent Chapter 6.

\vskip.5\baselineskip\hrule\vskip.5\baselineskip



\end
