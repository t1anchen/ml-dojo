%% \tracingall
\input manmac
\hsize=6.5in
\vsize=8.9in
\def\myhr{\vskip.5\baselineskip\hrule\vskip.5\baselineskip}


\noindent Chapter 1
\myhr
\ansno1.1:
 A \TeX nician (underpaid); sometimes also called a \TeX acker.
\myhr
\vfill
\eject


\noindent Chapter 2

\myhr
%% [2025-05-06T18:14:46+08:00]
``I understand''

%% [2025-05-06T18:12:00+08:00] Do not write comment inside the block
%% between `\begintt' and `\endtt' because the anything between
%% `\begintt' and `\endtt' will be eventually printed out in output
%% pdf.

%% [2025-05-06T18:16:22+08:00] `\]' represents "visible spacebar" in
%% normal/display mode
Visible spacebar: ``I\]understand.''

%% [2025-05-06T18:13:44+08:00] `|]' represents "visible spacebar" in
%% verbatim mode
\begintt
Verbatim spacebar: ``I|]understand.''
\endtt

\begindisplay
|-| & hyphen (-)\cr
|--| & en-dash (--)\cr
|---| & em-dash (---)\cr
|$-$| & minus sign $-$\cr
\enddisplay
\hrule

\ansno2.1: Alice said, ``I always use an en-dash instead of a hyphen
when specifying page numbers like `480--491' in a bibliography.''


\ansno2.2: Four hyphens `|----|' becomes one em-dash and one hyphen
`----'

\ansno2.3: fluff
\myhr
\noindent{\bf Summary}:
\item\bull ligatures
\item\bull kerning
\item\bull alternative input methods for left quotes, right quotes

\begindisplay
{\tt |\lq\lq\]I\]understand.\rq\rq|} $\Rightarrow$ \lq\lq\]I\]understand.\rq\rq
\enddisplay
\hrule


\ansno2.4: ``\thinspace` and `\thinspace``


\ansno2.5: Following Occam's Razor Principle, introducing extra
complexity on typewriting annotations might not be a wise move because
it increases extra maintenance cost in minds and memory. Besides,
introducing dollar sign might cause confusions between {\sl Math mode}
and {\sl Text mode} that it may deteriorate maintenability of code.
\myhr
\vfill
\eject


\noindent Chapter 3.
\myhr

Note: |ESC| key $\neq$ escape character

\noindent{\bf Summary}:
\item\bull control sequence, control symbol

\myhr
\ansno3.1:
`|\I'm \exercise3.1\\!|' control sequences are `|\I|', `|\exercise|',
 and `|\\|'

\ansno3.2:
\begindisplay
`math\'ematique' & $\rightarrow$\quad |`math\'ematique'|\cr
`centim\`etre' & $\rightarrow$\quad |`centim\`etre'|\cr
\enddisplay
\myhr
note: \TeX\ ignores spaces after control worlds

\noindent{\bf Summary}:
\item\bull apprimately 300 of 900 control sequence in \TeX\ are {\sl primitive}
\myhr
\ansno3.3: |\|\] is primitive, while |\|\<return> is represented as |^^M|

\ansno3.4:
([2025-05-09T00:11:26+08:00] after asking Perplexity.AI about how to
understand Knuth's answer, it makes sense now)

For length~2 case, the first position is always occupied by `|\|', the
remaining position is for a single ASCII character, which is 256
different variants (in the problem description, it said `including
escape character').

For length~3, it might be strictly limited to alphabet characters
(i.e. |[A-Za-z]| if represented in Regex). The first position is
contantly as `|\|', and each of the remaining positions has 52
variants, which is $52^2$ possible control sequences.
\myhr

\vfill
\eject


\noindent Chapter 4.
\myhr
\begindisplay
|\rm| & {\rm Roman}\cr
|\sl| & {\sl Slanted}\cr
|\it| & {\it Italic}\cr
|\tt| & {\tt Typewriter}\cr
|\bf| & {\bf Bold}\cr
\enddisplay
\myhr

\ansno4.1:
`Ulrich Dieter, {\sl Journal f\"ur die reine und angewandte
Mathematik} {\bf 201} (1959), 37--70.'

\ansno4.2:
{\it Explain how to typeset a\/ {\rm roman} word in the midst of an
italicized sentence.}

\ansno4.3:
|\def\ic#1{\setbox0=\hbox{#1\/}\dimen0=\wd0|\parbreak
|\setbox0=\hbox{#1}\advance\dimen0 by -\wd0 }|.
\myhr
\tenrm smaller \ninerm and smaller
\eightrm and smaller \sevenrm and smaller
\sixrm and smaller \fiverm and smaller \tenrm
\myhr
\ansno4.4:
The numeric ones would be possibly interpreted as `|\1|' and `0point'
instead of the entire control sequence.

\ansno4.5:
Quoted from official answer: ``Say |\def\sl{\it}| at the beginning,
and delete other definitions of\/ |\sl| that might be present in your
format file (e.g., there might be one inside a |\tenpoint| macro).''

%% \myhr

%% [2025-05-12T09:43:23+08:00] If it's ttf/otf font, it is required to
%% be preprocessed by mktexmf first, otherwise an error will be thrown
%% out
%%
%%     kpathsea: Running mktextfm Cambria
%%
%%     The command name is $TEXLIVE_HOME\2024\bin\windows\mktextfm
%%
%%     kpathsea: Running mktexmf Cambria.mf
%%
%%     The command name is $TEXLIVE_HOME\2024\bin\windows\mktexmf
%%     name = Cambria, rootname = Cambria, pointsize =
%%     mktexmf: empty or non-existent rootfile!
%%     Cannot find Cambria.mf.
%%     kpathsea: Appending font creation commands to build/missfont.log.
%%
%% ----
%% \font\ninerm=Cambria at 10pt

%% \myhr

\ansno4.6:
|\font\tenrm=cmr10 at 5pt|\parbreak
|\font\tenrm=cmr10 scaled 500|

Note: In the official answer, it uses `|\squinttenrm|'
\myhr
\font\msb=msbm10 at 10pt
\def\ldots{\mathinner{\ldotp\ldotp\ldotp}}

\noindent{\bf Summary}:
\item\bull magnification ratio factor
\item\bull $1.2^n, n \in $\thinspace {\msb N}
\item\bull |\magstep0|, |\magstep1|, $\ldots$, |\magstep5|
\myhr
\vfill
\eject


\noindent Chapter 5.
\myhr
\noindent{\bf Summary}:
\item\bull Use `|{|', `|}|' for grouping
\myhr
\ansno5.1:
Use grouping: `|shelf{f}ul|'\ $\rightarrow$\ `shelf{f}ul' rather than `shelfful'\parbreak

Official answer: |{shelf}ful| or |shelf{}ful|, etc.; or even
|shelf\/ful|, which yields a shelf\/ful instead of a
shelf{\kern0pt}ful.

\ansno5.2:
Use grouping: `|{\]}{\]}{\]}|' $\rightarrow$ `{\]}{\]}{\]}'\parbreak

Official answer: `\]|{|\]|}|\]' or `\]|{}|\]|{}|\]', etc. Plain \TeX\
also has a |\space| macro, so you can type |\space\space\space|.  \
(These aren't strictly equivalent to `|\|\]|\|\]|\|\]', since they
adjust the spaces by the current ``{space factor},'' as explained
later.)
\myhr
\noindent{\bf Summary}:

\item\bull |\centerline|
\item\bull grouping
\item\bull scope
\item\bull nested

Example:
\begindisplay
\centerline{This information should be {\it centered}.}
\enddisplay
\myhr
\ansno5.3:
First case: same\parbreak
second case: Only `S' centred but rest would be automatically arranged.

\ansno5.4:
The line will be centred and `centered' will be italic.

\ansno5.5:
\def\ital#1{{\it #1\/}}\parbreak
|\def\ital#1{{\it #1\/}}|

|\ital{hello} world| $\rightarrow$ \ital{hello} world

\item\bull Pros: scoped, more legible
\item\bull Cons: learning cost

%% \global\advance\count0 by 1

\ansno5.6:
Official answer: |{1 {2 3 4 5} 4 6} 4|

%% \def\c#1{\count1=#1}
%% \def\g{\global\count1=}
%% \def\s{\showthe\count1}

%% \hbox{%
%%   \c1\s
%%   \g2
%%   {%
%%     \s
%%     \c3\s
%%     \g4\s
%%     \c5\s
%%   }%
%%   \s
%%   \c6\s
%% }%
%% \s

%% ===================================================================
%% [2025-05-13T19:40:35+08:00] Cannot reproduce the result neither in
%% TeX nor in pdfTeX. It keeps alarming errors like
%%
%%     <to be read again>
%%                        \global
%%     \g ->\global
%%                  \count 1=
%%     l.268   \g
%%               2

%% it does not work even braces added/removed or marco parameter
%% added/removed. It seems like the behaviorial differences between
%% different versions of TeX.
\myhr
\vfill
\eject

\noindent Chapter 6.
\myhr
$$\halign{\hbox to\parindent{\hfil\sevenrm#\ \ }&#\hfil\cr
1&|\hrule|\cr
2&|\vskip 1in|\cr
3&|\centerline{\bf A SHORT STORY}|\cr
4&|\vskip 6pt|\cr
5&|\centerline{\sl by A. U. Thor}|\cr
6&|\vskip .5cm|\cr
7&|Once upon a time, in a distant|\cr
8&|  galaxy called \"O\"o\c c,|\cr
9&|there lived a computer|\cr
10&|named R.~J. Drofnats.|\cr
11&||\cr
12&|Mr.~Drofnats---or ``R. J.,'' as|\cr
13&|he preferred to be called---|\cr
14&|was happiest when he was at work|\cr
15&|typesetting beautiful documents.|\cr
16&|\vskip 1in|\cr
17&|\hrule|\cr
18&|\vfill\eject|\cr}$$

\vskip.5\baselineskip

\hrule
\vskip 1in
\centerline{\bf A SHORT STORY}
\vskip 6pt
\centerline{\sl by A. U. Thor}
\vskip .5cm
Once upon a time, in a distant
  galaxy called \"O\"o\c c,
there lived a computer
named R.~J. Drofnats.

Mr.~Drofnats---or ``R. J.,'' as
he preferred to be called---
was happiest when he was at work
typesetting beautiful documents.
\vskip 1in
\hrule
\vfill\eject
\myhr
\noindent{\bf Summary}:

\item\bull `|\c|' (cedilla)
\item\bull `|~|' (ties, treat as normal space but not
to break between lines, e.g. names)
\myhr
\ansno6.1:
Official answer: ``Laziness and/or obstinacy''

\ansno6.2:
Official answer: `` There's an unwanted space after `called---',
because (as the book says) \TeX\ treats the end of a line as if it
were a blank space. That blank space is usually what you want, except
when a line ends with a hyphen or a dash; so you should WATCH OUT for
lines that end with hyphens or dashes''

\ansno6.3:
Official answer: ``It represents the heavy bar that shows up in your
output. \ (This bar wouldn't be present if had been set to |0pt|, nor
is it present in an underfull box.)''

\ansno6.4:
Official answer: ``This is the |\parfillskip| space that ends the
paragraph.  In plain \TeX\ the parfillskip is zero when the last line
of the paragraph is full; hence no space actually appears before the
rule in the output of Experiment~3. But all hskips show up as spaces
in an overfull box message, even if they're zero.''

\ansno6.5:
Official answer: ``Run \TeX\ with \hbox{|\hsize=1.5in|} \hbox{|\tolerance=10000|}
\hbox{|\raggedright|} \hbox{|\hbadness=-1|} and then |\input story|. \TeX\ will
report the badness of all lines (except the final lines of paragraphs,
where fill glue makes the badness zero).''

\ansno6.6:
Official answer: ``|\def\extraspace{\nobreak \hskip 0pt plus
.15em\relax}|\parbreak
|\def\dash{\unskip\extraspace---\extraspace}|\par\nobreak\smallskip\noindent
(If you try this with the story at 2-inch and 1.5-inch sizes, you will
notice a substantial improvement. The |\unskip| allows people to leave
a space before typing |\dash|.  \TeX\ will try to hyphenate before
|\dash|, but not before `|---|'; cf.\ Appendix~H\null. The |\relax|
at the end of |\extraspace| is a precaution in case the next word is
`|minus|'.)''

\ansno6.7:
Official answer: ``\TeX\ would have deleted five tokens: |1|, |i|,
|n|, \], |\centerline|.  (The space was at the end of line~2, the
|\centerline| at the beginning of line~3.)''

\ansno6.8:
Official answer: ``A control sequence like |\centerline| might well
define a control sequence like |\ERROR| before telling \TeX\ to look
at |#1|. Therefore
\TeX\ doesn't interpret control sequences when it scans an argument.''

\myhr
\vfill
\eject


\noindent Chapter 7.
\myhr

\noindent{\bf Summary}:
\item\bull Reserved $10$ characters (must be escaped with `|\|'): `|\|', `|{|', `|}|', `|$|', `|&|', `|#|', `|^|', `|_|', `|%|', `|~|'
\item\bull example: `|$\{a \backslash b\}$|' $\rightarrow$ `$\{a \backslash b\}$'
\myhr
\ansno7.1:
The `|&|', `|$|' and `|%|' must be escaped.

\ansno7.2:
Official answer: ``Reverse slashes (backslashes) are fairly uncommon
in formulas or text, and |\\| is very easy to type; it was therefore
felt best not to reserve |\\| for such limited use. Typists can define
|\\| to be whatever they want (including |\backslash|).''
\myhr
\noindent{\bf Summary}:
\item\bull \cstok{boxed token}
\myhr
\ansno7.3:
Official answer: ``1, 2, 3, 4, 6, 7, 8, 10, 11, 12, 13. {Active
characters} (type 13) are somewhat special; they behave like control
sequences in most cases (e.g., when you say `|\let||\x=~|' or
`|\ifx||\x~|'), but they behave like character tokens when they appear
in the token list of\/ |\uppercase| or |\lowercase|, and when
unexpanded after |\if| or |\ifcat|.''

\ansno7.4:
Official answer: ``It ends with either |>| or |}| or any character of
category 2; then the effects of all |\catcode| definitions within the
group are wiped out, except those that were |\global|.''

\ansno7.5:
Official answer: ``If you type `|\message{\string~}|' and `|\message{\string\~}|', \TeX\
responds with `|~|' and `|\~|', respectively. |\message|
To get |\|$_{12}$ from |\string| you therefore need to make backslash an
active character. One way to do this is
\begintt
{\catcode`/=0 \catcode`\\=13 /message{/string\}}
\endtt
(The ``{null control sequence}'' that you get when there are no
tokens between |\csname| and |\endcsname| is not a solution to this exercise,
because |\string| converts it to `|\csname\endcsname|'. There is, however,
another solution: If \TeX's |\escapechar| parameter---which will be
explained in one of the next dangerous bends---is negative or greater
than~255, then `|\string\\|' works.)''

\ansno7.6:
|\|$_{12}$ |a|$_{12}$ |\|$_{12}$ \]$_{10}$ |b|$_{12}$.

\ansno7.7:
Official answer: ``|\def\ifundefined#1{\expandafter\ifx\csname#1\endcsname\relax}|%
\hfil\break Note that a control sequence like this must be used with care;
it cannot be included in {conditional} text, because the |\ifx| will not
be seen when |\ifundefined| isn't expanded.''

\ansno7.8:
|\uppercase{a\lowercase{bC}}| $\rightarrow$ `\uppercase{a\lowercase{bC}}'

\ansno7.9:
`\copyright\ \uppercase\expandafter{\romannumeral\year}' $\rightarrow$ `|\copyright\ \uppercase\expandafter{\romannumeral\year}|'

\ansno7.10:
Official answer:
\begintt
\def\gobble#1{} % remove one token
\def\appendroman#1#2#3{\expandafter\def\expandafter#1\expandafter
  {\csname\expandafter\gobble\string#2\romannumeral#3\endcsname}}
\endtt
\myhr
\vfill
\eject


\noindent Chapter 8.
\myhr
\noindent{\bf Summary}:

\item\bull `|\char98 u\char98\char98 le|' $\rightarrow$ `\char98 u\char98\char98 le'

\halign{\indent\hfil# && $\rightarrow$ #\cr
`|\char98|' & `\char98'\cr
`|\char'142|' & `\char'142'\cr
`|\char"62|' & `\char"62'\cr
}

\myhr
\ansno8.1:

|%| in \TeX can be interpreted as reserved keyword of comment
|%| character.

Official answer: ``The |%| would be treated as a comment character,
because its category code is~14; thus, no |%| token or |}| token would
get through to the gullet of \TeX\ where numbers are treated. When a
character is of category 0, 5, 9, 14, or~15, the extra |\| must be
used; and the |\| doesn't hurt, so you can always use it to be safe.''

\ansno8.2:
\halign{\indent\hfil#\ &\ #\hfil\cr
(a)&A caracter of category~5 can be converted into `\]'$_{10}$ or
a \cstok{par} token, \cr
&while a character of category~14 never produces a token.\cr
(b)&Different category numbers\cr
(c)&Same as (b)\cr
(d)&No\cr
(e)&Yes\cr
(f)&No\cr
}
\myhr
\ansno8.3: TBD
\ansno8.4: TBD
\ansno8.5: TBD
\ansno8.6: TBD
\ansno8.7: TBD

\myhr
\vfill
\eject


\noindent Chapter 9.
\myhr

\noindent{\bf Summary}:
\item\bull upper accent

$$\halign{\indent\hbox to 50pt{#\hfil}&\hbox to 35pt{#\hfil}&#\hfil\cr
\it\negthinspace Type&\it to get\cr
\noalign{\smallskip}
|\`o|&\`o&(grave accent)\cr
|\'o|&\'o&(acute accent)\cr
|\^o|&\^o&(circumflex or ``hat'')\cr
|\"o|&\"o&(umlaut or dieresis)\cr
|\~o|&\~o&(tilde or ``squiggle'')\cr
|\=o|&\=o&(macron or ``bar'')\cr
|\.o|&\.o&(dot accent)\cr
|\u o|&\u o&(breve accent)\cr
|\v o|&\v o&(h\'a\v cek or ``check'')\cr
|\H o|&\H o&(long Hungarian umlaut)\cr
|\t oo|&\t oo&(tie-after accent)\cr}$$

accents that go underneath

$$\halign{\indent\hbox to 50pt{#\hfil}&\hbox to 35pt{#\hfil}&#\hfil\cr
\it\negthinspace Type&\it to get\cr
\noalign{\smallskip}
|\c o|&\c o&(cedilla accent)\cr
|\d o|&\d o&(dot-under accent)\cr
|\b o|&\b o&(bar-under accent)\cr}$$

Special letters

$$\halign{\indent\hbox to 50pt{#\hfil}&\hbox to 35pt{#\hfil}&#\hfil\cr
\it\negthinspace Type&\it to get\cr
\noalign{\smallskip}
|\oe,\OE|&\oe,\thinspace\OE&(French ligature OE)\cr
|\ae,\AE|&\ae,\thinspace\AE&(Latin ligature and Scandinavian letter AE)\cr
|\aa,\AA|&\aa,\thinspace\AA&(Scandinavian A-with-circle)\cr
|\o,\O|&\o,\thinspace\O&(Scandinavian O-with-slash)\cr
|\l,\L|&\l,\thinspace\L&(Polish suppressed-L)\cr
|\ss|&\ss&(German ``es-zet'' or sharp S)\cr}$$
\myhr
\ansno9.1: `|na\"\i ve|' $\rightarrow$ `na\"\i ve'

\ansno9.2:
Official answer: ``Belov\`ed prot\'eg\'e; r\^ole co\"ordinator;
souffl\'es, cr\^epes, p\^at\'es, etc.''

\ansno9.3:
`|\AE sop's \OE uvres en fran\c cais|' $\rightarrow$ `\AE sop's \OE
uvres en fran\c cais'

\ansno9.4:
`|{\sl Commentarii Academi\ae\ scientiarum imperialis|\hfil\break
|petropolitan\ae\/} became {\sl Akademi\t\i a Nauk SSSR,| \hfil\break
|Doklady}|' $\rightarrow$ `{\sl Commentarii Academi\ae\ scientiarum
imperialis petropolitan\ae\/} became {\sl Akademi\t\i a Nauk SSSR,
Doklady}'

\ansno9.5:
\halign{\indent\hfil#$\rightarrow$&\ #\hfil\cr
`|Ernesto {Ces\`aro}|'                           & `Ernesto {Ces\`aro}'\cr
`|P\'al {Erd\H os}|'                             & `P\'al {Erd\H os}'\cr
`|\O ystein {Ore}|'                              & `\O ystein {Ore}'\cr
`|Stanis\l aw \'Swierczkowski|'                  & `Stanis\l aw \'Swierczkowski' \cr
`|Serge\u\i\ \t Iur'ev|'                         & `Serge\u\i\ \t Iur'ev'\cr
`|Mu\d hammad ibn M\^us\^a {al-Khw\^arizm\^\i}|' & `Mu\d hammad ibn M\^us\^a {al-Khw\^arizm\^\i}'\cr
}

\ansno9.6:
`|{\tt P\'al Erd{\bf\H{\tt o}}s}|' $\rightarrow$ `{\tt P\'al Erd{\bf\H{\tt o}}s}'
\myhr
\noindent{\bf Summary}:
\item\bull Special signs

$$\halign{\indent#\hfil\ &\hfil#\hfil&#\hfil\cr
\it\negthinspace Type&\it to get\cr
\noalign{\smallskip}
|\dag|&\dag&(dagger or obelisk)\cr
|\ddag|&\ddag&(double dagger or diesis)\cr
|\S|&\S&(section number sign)\cr
|\P|&\P&(paragraph sign or pilcrow)\cr}$$
\myhr
\ansno9.7:
`|{\it Europe on {\sl\$}15.00 a day\/}|' $\rightarrow$ `{\it Europe on
{\sl\$}15.00 a day\/}'

\ansno9.8:
Official answer: ``The extra braces keep font changes local. An
argument makes the use of\/ |\'| more consistent with the use of other
accents like |\d|, which are manufactured from other characters
without using the |\accent| primitive.''
\myhr
\vfill
\eject


\noindent Chapter 10.

\myhr
\noindent{\bf Summary}
\item\bull Units

$$\halign{\indent\tt#&\quad#\hfil\cr
pt&point (baselines in this manual are $12\pt$ apart)\cr
pc&pica ($\rm1\,pc=12\,pt$)\cr
in&inch ($\rm1\,in=72.27\,pt$)\cr
bp&big point ($\rm72\,bp=1\,in$)\cr
cm&centimeter ($\rm2.54\,cm=1\,in$)\cr
mm&millimeter ($\rm10\,mm=1\,cm$)\cr
dd&didot point ($\rm1157\,dd=1238\,pt$)\cr
cc&cicero ($\rm1\,cc=12\,dd$)\cr
sp&scaled point ($\rm65536\,sp=1\,pt$)\cr}$$
\myhr
\ansno10.1:
254 centimeters $=$ 100 in $=$ 7227 pt
\myhr
\noindent{\bf Summary}:
\item\bull Parsing syntax

$$\halign{\indent\hfil#&&\quad #\hfil\cr
Syntax 1:&\<optional sign>\<number>\<unit of measure>\cr
Syntax 2:&\<optional sign>\<digit string>|.|\<digit string>\<unit of measure>\cr
}$$
\myhr
\ansno10.2:
\halign{\indent\hfil#\ &\hfil#&\thinspace #\hfil\ &\ #\cr
1 & -.013837  & in & (-0.3514598 mm)\cr
2 & 0.        & mm & (0.0 mm)\cr
3 & +42.1     & dd & (15.832338 mm)\cr
4 & 3         & in & (76.2 mm)\cr
5 & 29        & pc & (122.308012 mm)\cr
6 & 123456789 & sp & (662.080365 mm)\cr
}

\ansno10.3:
\halign{\indent\hfil#\ &\ #\hfil\cr
|'.77pt| & Not legitimate \cr
|"Ccc| & $\rightarrow 12$ cc\cr
|-,sp| & $\rightarrow 0.0$ sp\cr
}

\ansno10.4:
Official answer: ``
% tc: tianchen, tpf: ten point four, bs: backslash
\begintt
\def\tick#1{\vrule height 0pt depth #1pt}
\def\tctpfbs{\hbox to 1cm{\hfil\tick4\hfil\tick8}}
\vbox{\hrule\hbox{\tick8\tctpfbs\tctpfbs\tctpfbs\tctpfbs
\tctpfbs\tctpfbs\tctpfbs\tctpfbs\tctpfbs\tctpfbs}}
\endtt''

\indent\vbox{
\def\tick#1{\vrule height 0pt depth #1pt}
\def\tctpfbs{\hbox to 1cm{\hfil\tick4\hfil\tick8}}
\vbox{\hrule\hbox{\tick8\tctpfbs\tctpfbs\tctpfbs\tctpfbs
\tctpfbs\tctpfbs\tctpfbs\tctpfbs\tctpfbs\tctpfbs}}
}
\myhr
\noindent{\bf Summary}:

\halign{\indent\hfil#\ &\ #\hfil\cr
|\magnification=1200| & Enlarge 1.2 times normal size\cr
|\magnification=2000| & Double\cr
}
\myhr
\ansno10.5:

\vbox{
1.2 times\vskip5pt
\font\enlargedrm=cmr10 scaled\magstep1
\enlargedrm\input story
}
\vbox{
\vskip5pt 1.44 times\vskip5pt
\font\enlargedrm=cmr10 scaled\magstep2
\enlargedrm\input story
}
\vbox{
\vskip5pt 1.728 times\vskip5pt
\font\enlargedrm=cmr10 scaled\magstep3
\enlargedrm\input story
}
\myhr

\noindent{\bf Summary}:
\item\bull |\hsize=6.5 true in| represents no-scaled 6.5 inch for text area height

\myhr
\ansno10.6:
%% [2025-05-22T17:27:24+08:00] `\magnification' Not work in pdftex
Official answer: ``Magnification is by a factor of 1.2. Since font |\first| is |cmr10|
at $12\pt$, it will be |cmr10| at $14.4\pt$ after magnification;
font |\second| will be |cmr10| at $12\pt$. \ (\TeX\ changes
`|12truept|' into `|10pt|', and the final output magnifies it back to
$12\pt$.)''
\myhr
\vfill
\eject


\noindent Chapter 11.
\myhr
\noindent{\bf Summary}:
\item\bull black box: `\vrule width 4pt height 6pt depth 1.5pt', `\bull'

\vbox{\hbox{Two lines}\hbox{of type.}}
\myhr

\ansno11.1:
The `|E|' is inside a box that is inside a box.

\ansno11.2:
Official answer: ``The idea is to construct a box and to look inside. For example,
\begintt
\setbox0=\hbox{\sl g\/} \showbox0
\endtt
reveals that |\/| is implemented by placing a kern after the character.
Further experiment shows that this kern is inserted even when the italic
correction is zero.''

\ansno11.3:
Official answer: ``The height, depth, and width of the enclosing box should be just large
enough to enclose all of the contents, so the result is:
\begintt
\hbox(8.98608+0.0)x24.44484
.\tenrm T
.\kern 1.66702
.\hbox(6.83331+0.0)x6.80557, shifted -2.15277
..\tenrm E
.\kern 1.25
.\tenrm X
\endtt
(You probably predicted a height of |8.9861|; \TeX's internal calculations are
in |sp|, not |pt|/100000, so the rounding in the fifth decimal place is not
readily predictable.)''

\ansno11.4:
\TeX\ is supposed to be designed for English-language world. For a general,
symmetrical box design, {\it CSS3 Box Model} might be a fair implementation
for reference.

\ansno11.5:
\vbox{
\def\dolist{\afterassignment\dodolist\let\next= }
\def\dodolist{\ifx\next\endlist \let\next\relax
  \else \\\let\next\dolist \fi
  \next}
\def\endlist{\endlist}
\def\hidehrule#1#2{\kern-#1%
  \hrule height#1 depth#2 \kern-#2 }
\def\hidevrule#1#2{\kern-#1{\dimen0=#1
    \advance\dimen0 by#2\vrule width\dimen0}\kern-#2 }
\def\makeblankbox#1#2{\hbox{\lower\dp0\vbox{\hidehrule{#1}{#2}%
    \kern-#1 % overlap the rules at the corners
    \hbox to \wd0{\hidevrule{#1}{#2}%
      \raise\ht0\vbox to #1{}% set the vrule height
      \lower\dp0\vtop to #1{}% set the vrule depth
      \hfil\hidevrule{#2}{#1}}%
    \kern-#1\hidehrule{#2}{#1}}}}
\def\maketypebox{\makeblankbox{0pt}{1pt}}
\def\makelightbox{\makeblankbox{.2pt}{.2pt}}
\def\\{\if\space\next\ % assume that \next is unexpandable
 \else \setbox0=\hbox{\next}\maketypebox\fi}
\def\demobox#1{\setbox0=\hbox{\dolist#1\endlist}%
  \leavevmode\copy0\kern-\wd0\makelightbox}


\noindent|\demobox{Tough exercise.}| $\rightarrow$ \demobox{Tough exercise.}
}

\ansno11.6:
\vbox{
\def\frac#1/#2{\leavevmode\kern.1em
  \raise.5ex\hbox{\the\scriptfont0 #1}\kern-.1em
  /\kern-.15em
  \lower.25ex\hbox{\the\scriptfont0 #2}}

\noindent|\frac1/2| $\rightarrow$ `\frac1/2'
}

\myhr
\vfill
\eject


\noindent Chapter 12.
\myhr
\noindent{\bf Summary}:
\item\bull box with glue example

\begindisplay\eightpoint
\vbox{
    \hbox{\samplebox{7mm}{8mm}{5\varunit}{width 5}%
      \sampleglue{9\varunit}{space 9\cr stretch 3\cr shrink 1}%
      \samplebox{3mm}{2mm}{6\varunit}{width 6}%
      \sampleglue{9\varunit}{space 9\cr stretch 6\cr shrink 2}%
      \samplebox{8mm}{3mm}{3\varunit}{width 3}%
      \sampleglue{12\varunit}{space 12\cr stretch 0\cr shrink 0}%
      \samplebox{4mm}{7mm}{8\varunit}{width 8}}
    \kern6pt
    \arrows{52\varunit}{width 52}}
\enddisplay

\item\bull glue stretch/shrink arithmetics

In the example above, if the total width is expected to be $58$, then
\vskip\baselineskip
\halign{\indent\hfil#\ &\ #\hfil\cr
1 & Calculate how many units are require to be stretched/shrinked\cr
& i.e. $58-52=6$\cr
2 & formula: natural $\pm$ (total actual needed $\div$ total capability $\times$ actual needed)\cr
& i.e. $9 + (6/9)\times 3 = 11$, $9 + (6/9)\times 6 = 13$, $12 + (6/9)\times 0 = 12$\cr
& i.e. $5 + 11 + 6 + 13 + 3 + 12 + 8 = 58$\cr
}

\item\bull setting the glue (once set the glue becomes rigid)
\item\bull glue cannot shrink more than shrinkability, but can stretch arbitraily far

\myhr
\ansno12.1:
\halign{\indent\hfil#\ &\ #\hfil\cr
Step 1.&$100-52=48$ globs require to be added/stretched\cr
Step 2.&$9 + (48/9)\times 3 = 9+16$, $9 + (48/9)\times 6 = 9+32$, $9 + (48/9)\times 0 = 12+0$\cr
}

Since $100$ exceeds the maximal stretchable width, \TeX\ will consider
as ``infinitely bad''.
\myhr
\noindent{\bf Summary}:
\item\bull glue syntax: |\<glue_control_sequence><dimen> plus<dimen> minus<dimen>|
\item\bull |\vfil| vs |\vfill|
\vskip1em
\halign{\indent\hfil#\ &\ #\hfil\cr
|\vfil| & Expand to fill the remaining space\cr
|\vfill| & effectively prevents |\vfil| from stretching\cr
|\hfil| & horizontal fill\cr
|\hfill| & horizontal fill (stronger)\cr
|\hss| & horizontal stretch or shrink\cr
|\vss| & vertical stretch or shrink\cr
|\hfilneg| & cancel |\hfil|\cr
|\vfilneg| & cancel |\vfil|\cr
}
\vskip1em
\item\bull |\hfil| examples
\vskip\baselineskip

\def\borderedbox#1{
\vbox{\hrule
\hbox{\vrule\kern1em\vbox{
\parindent=0pt
\vskip1em #1\vskip1em
}
\kern1em\vrule}
\hrule}}

%% [2025-05-28T09:33:22+08:00] it cannot use borderedbox macro below
%% due to the verbatim content wrapped by `|' would be expanded
%% automatically in a macro and there is no valid approach to mitigate
%% this issue.

\vbox{\hrule
\hbox{\vrule\kern1em\vbox{
\parindent=0pt
\vskip1em
\line{|\line{This text will be flush left\hfil}|}
\line{This text will be flush left.\hfil}
\vskip1em
}
\kern1em\vrule}
\hrule}
\vskip-1pt
\vbox{
\hbox{\vrule\kern1em\vbox{
\parindent=0pt
\vskip1em
\line{|\line{\hfil This text will be flush right.}|}
\line{\hfil This text will be flush right.}
\vskip1em
}
\kern1em\vrule}
\hrule}
\vskip-1pt
\vbox{
\hbox{\vrule\kern1em\vbox{
\parindent=0pt
\vskip1em
\line{|\line{\hfil This text will be centered.\hfil}|}
\line{\hfil This text will be centered.\hfil}
\vskip1em
}
\kern1em\vrule}
\hrule}
\vskip-1pt
\vbox{\hrule
\hbox{\vrule\kern1em\vbox{
\parindent=0pt
\vskip1em
\line{|\line{Some text flush left\hfil and some flush right.}|}
\line{Some text flush left\hfil and some flush right.}
\vskip1em
}
\kern1em\vrule}
\hrule}
\vskip-1pt
\vbox{
\hbox{\vrule\kern1em\vbox{
\parindent=0pt
\vskip1em
\line{|\line{Alpha\hfil centered between Alpha and Omega\hfil Omega}|}
\line{Alpha\hfil centered between Alpha and Omega\hfil Omega}
\vskip1em
}
\kern1em\vrule}
\hrule}
\vskip-1pt
\vbox{
\hbox{\vrule\kern1em\vbox{
\parindent=0pt
\vskip1em
\line{|\line{Five\hfil words\hfil equally\hfil spaced\hfil out.}|}
\line{Five\hfil words\hfil equally\hfil spaced\hfil out.}
\vskip1em
}
\kern1em\vrule}
\hrule}

\vskip\baselineskip

\myhr

\ansno12.2:
Official answer: `` |\line{\hfil\hfil What happens now?\hfil}| is
placed in a line of width\break |\hsize|, with twice as much space at the
left as at the right.''
\vskip.5em
\borderedbox{
\line{\hfil\hfil What happens now?\hfil}
}

`` |\line{\hfill\hfil and now?\hfil}|\ is put flush right on the
following line.''
\vskip.5em
\borderedbox{
\line{\hfill\hfil and now?\hfil}
}

\ansno12.3:
Official answer: ``The first two give an ``overfull box'' if the
argument doesn't fit on a line; the third allows the argument to stick
out into the margins instead. \ (Plain \TeX's ^|\centerline| is
|\centerlinec|; the stickout effect shows up in the narrow-column
experiment of Chapter~6.) \ If the argument contains no infinite glue,
|\centerlinea| and |\centerlineb| produce the same effect; but
|\centerlineb| will center an argument that contains `fil' glue.''

\myhr
\ansno12.4:

\noindent``|Mr.~\& Mrs.~User were married by Rev.~Drofnats,|\break
|who preached on Matt.~19\thinspace:\thinspace3--9.|'' $\rightarrow$\hfil\break
``Mr.~\& Mrs.~User were married by Rev.~Drofnats, who preached on
Matt.~19\thinspace:\thinspace3--9.''

\ansno12.5:
|Donald~E.\ Knuth, ``Mathematical typography,'' {\sl Bull.\
Amer.\ Math.\ Soc.\ \bf1} (1979), 337--372.|
``Donald~E.\ Knuth, ``Mathematical typography,'' {\sl Bull.\
Amer.\ Math.\ Soc.\ \bf1} (1979), 337--372.''

\ansno12.6:
Official answer: ``There are several ways; perhaps the easiest are to
type `|\hbox{NASA}.|'\ or `|NASA\null.|' \ (The ^|\null| macro is an
abbreviation for `|\hbox{}|'.)''

%% [2025-05-30T21:00:50+08:00] Tired of being perceiving from
%% AI. Wasting life ...

\ansno12.7:
Official answer: ``1000, except: 999 after |O|, |B|, |S|, |D|, and
|J|; 1250 after the comma; 3000 after the exclamation point, the
right-quote marks, and the periods. If a period had come just after
the |B| (i.e., if the text had said `|B. Sally|'), the space factor
after that period would have been~1000, not~3000.''

\ansno12.8:
Official answer: ``|\box3| is $2\pt$ high, $4\pt$ deep, $3\pt$ wide.
Starting at the reference point of\/ |\box3|, go right $.75\pt$ and
down $3\pt$ to reach the reference point of\/ |\box1|; or go right
$1\pt$ to reach the reference point of\/ |\box2|.''

\ansno12.9:
Official answer: ``The stretch and shrink components of\/
|\baselineskip| and |\lineskip| should be equal, and the
|\lineskiplimit| should equal the normal |\lineskip| spacing, to
guarantee continuity.''

\ansno12.10:
Official answer: ``Yes it did, but only because none of his boxes had
a negative height or depth. He would have been safer if he had set
|\baselineskip=-1000pt|, |\lineskip=0pt|, and
|\lineskiplimit=16383pt|. \ (Plain \TeX's ^|\offinterlineskip| macro
does this.)''

\ansno12.11:
Official answer: ``The interline glue will be zero, and the natural
height is $1+1-3+2=1\pt$ (because the depth of\/ |\box2| isn't
included in the natural height); so the glue will ultimately become
|\vskip-1pt| when it's set.  Thus, |\box3| is $3\pt$ high, $2\pt$
deep, $4\pt$ wide. Its reference point coincides with that of\/
|\box2|; to get to the reference point of\/ |\box1| you go up $2\pt$
and right $3\pt$.''

\ansno12.12:
Official answer: ``The interline glue will be $6\pt$ minus $3\,{\rm fil}$; the final
depth will be zero, since |\box2| is followed by glue; the natural
height is $12\pt$; and the shrinkability is $5\,{\rm fil}$. So |\box4|
will be $4\pt$ high, $0\pt$ deep, $1\pt$ wide, and it will contain
five items: |\vskip|\penalty0\hbox{|-1.6pt|}, |\box1|, |\vskip1.2pt|,
|\moveleft4pt\box2|, |\vskip-1.6pt|. Starting at the reference point of
|\box4|, you get to the reference point of\/ |\box1| by going up $4.6\pt$,
or to the reference point of\/ |\box2| by going up $.4\pt$ and left $4\pt$.
\ (For example, you go up $4\pt$ to get to the upper left corner of
|\box4|; then down $-1.6\pt$, i.e., up $1.6\pt$, to get to the upper
left corner of\/ |\box1|; then down $1\pt$ to reach its reference
point.  This problem is clearly academic, since it's rather ridiculous
to include infinite shrinkability in the baselineskip.)''

\ansno12.13:
Official answer: ``Now |\box4| will be $4\pt$ high, $-4\pt$ deep,
$1\pt$ wide, and it will contain |\vskip|\penalty0\hbox{|-2.4pt|},
|\box1|, |\vskip-1.2pt|, |\moveleft4pt\box2|, |\vskip-2.4pt|. From the
baseline of\/ |\box4|, go up exactly $5.4\pt$ to reach the baseline
of\/ |\box1|, or exactly $3.6\pt$ to reach the baseline of\/
|\box2|.''

\ansno12.14:
Official answer: ``|\vbox to| $x$|{}| produces height $x$;
|\vtop to| $x$|{}| produces depth $x$; the other dimensions are zero.
\ (This holds even when $x$ is negative.)''

\ansno12.15:
Official answer: ``
There are several possibilities:
\begintt
\def\nullbox#1#2#3{\vbox to#1{\vss\hrule height-#2depth#2width#3}}
\endtt
works because the rule will be of zero thickness. Less tricky is
\begintt
\def\nullbox#1#2#3{\vbox to#1{\vss\vtop to#2{\vss\hbox to#3{}}}}
\endtt
Both of these are valid with negative height and/or depth, but they do
not produce negative width. If the width might be negative, but not the
height or depth, you can use, e.g.,
|\def\nullbox#1#2#3{\hbox to#3{\hss\raise#1\null\lower#2\null}}|.
It's impossible for |\hbox| to construct a box
whose height or depth is negative; it's impossible for |\vbox| or
|\vtop| to construct a box whose width is negative.\par
However, there's actually a trivial solution to the general problem, based on
features that will be discussed later:
\begintt
\def\nullbox#1#2#3{\setbox0=\null
  \ht0=#1 \dp0=#2 \wd0=#3 \box0 }
\endtt
''

\ansno12.16:
Official answer: ``|\def\llap#1{\hbox to 0pt{\hss#1}}|''

\ansno12.17:
Official answer: ``You get `A' at the extreme left and `puzzle.\null'
at the extreme right, because the space between words has the only
stretchability that is finite; the infinite stretchability cancels
out. \ (In this case, \TeX's rule about ^{infinite glue} differs from
what you would get in the limit if the value of $1\,{\rm fil}$ were
finite but getting larger and larger.  The true limiting behavior
would stretch the text `A~puzzle.\null' in the same way, but it would
also move that text infinitely far away beyond the right edge of the
page.)''

\myhr
\vfill
\eject


\noindent Chater 13.
\myhr
\noindent {\bf Summary}:
\halign{\indent\hfil#\ &\ #\hfil\cr
Vertial Mode               & Main Vertical List\cr
Internal Vertial Mode      & Vertical list for vbox\cr
Horizontal Mode            & horizontal list for a paragraph\cr
Restricted Horizontal Mode & horizontal list for an hbox\cr
Math Mode                  & math formula in horizontal list\cr
Display Math Mode          & math formula on a line by itself\cr
}

\myhr

\ansno13.1:
|\hbox| will not work

\ansno13.2:
Official answer: ``The output of\/ |\tracingcommands| shows that four
blank space tokens were digested; these originated at the ends of
lines 2,~3, 4, and~5. Only the first had any effect, since blank
spaces are ignored in math formulas and in vertical modes.''

\ansno13.3:
Official answer: ``The |end-group character| finishes the paragraph and the |\vbox|,
and |\bye| stands for `|\par\vfill...|', so the next three commands are
\begintt
{math mode: math shift character $}
{restricted horizontal mode: end-group character }}
{vertical mode: \par}
\endtt
''

\ansno13.4:
Official answer: ``It contains only mixtures of vertical glue and
horizontal rules whose reference points appear at the left of the
page; there's no text.''

\ansno13.5:
Official answer: ``Vertical mode can occur only as the outermost mode;
horizontal mode and display math mode can occur only when immediately
enclosed by vertical or internal vertical mode; ordinary math mode
cannot be immediately enclosed by vertical or internal vertical mode;
all other cases are possible.''

\myhr
\vfill
\eject


\noindent Chapter 14.
\myhr

\ansno14.1:
\halign{#\hfil\ &\ #\hfil\cr
|(cf.~Chapter~12).|                                &(cf.~Chapter~12).\cr
|Chapters 12 and~21.|                              &Chapters 12 and~21.\cr
|line~16 of Chapter~6's {\tt story}|               &line~16 of Chapter~6's {\tt story}\cr
|lines 7 to~11|                                    &lines 7 to~11\cr
|lines 2,~3, 4, and~5.|                            &lines 2,~3, 4, and~5.\cr
|(2)~a big black bar|                              &(2)~a big black bar\cr
|All 256~characters are initially of category~12,| &All 256~characters are initially of category~12,\cr
|letter~{\tt x} in family~1.|                      &letter~{\tt x} in family~1.\cr
|the factor~$f$, where $n$~is 1000~times~$f$.|     &the factor~$f$, where $n$~is 1000~times~$f$.\cr
}%

\ansno14.2:
`|for all $n$~greater than~$n_0$|' $\rightarrow$ `for all $n$~greater than~$n_0$'

\ansno14.3:
Official answer: `` `|exercise \hbox{4.3.2--15}|' guarantees that
there is no break after the ^{en-dash}. But this precaution is rarely
necessary, so `|exercise 4.3.2--15|' is an acceptable answer. No |~|
is needed; `\hbox{4.3.2--15}' is so long that it causes no offense at
the beginning of a line.''

\ansno14.4:
Official answer: ``The space you get from |~| will stretch or shrink
with the other spaces in the same line, but the space inside an hbox
has a fixed width since that glue has already been set once and for
all.  Furthermore the first alternative permits the word Chapter to be
^{hyphenate}d.''

\ansno14.5:
Official answer: ```|\hbox{$x=0$}|' is unbreakable, and we will see
later that `|${x=0}$|' cannot be broken. Both of these solutions set
the glue surrounding the equals sign to some fixed value, but such
glue normally wants to stretch; furthermore, the |\hbox| solution
might include undesirable blank space at the beginning or end of a
line, if\/ ^|\mathsurround| is nonzero. A third solution
`|$x=\nobreak0$|' avoids both defects.''

\ansno14.6:
Official answer: ``|\exhyphenpenalty=10000| prohibits all such breaks,
according to the rules found later in this chapter. Similarly,
|\hyphenpenalty=10000| prevents breaks after implicit (discretionary)
hyphens.''

\myhr

\noindent{\bf Summary}
{\obeylines\smallskip
Roses are red,
\quad Violets are blue;
Rhymes can be typeset
\quad With boxes and glue.
\smallskip}

\myhr

\ansno14.7:
Official answer: ``The second and fourth lines are indented by an additional ``quad''
of space, i.e., by one extra em in the current type style.
\ (The control sequence |\quad| does an ^|\hskip|; when \TeX\ is in
vertical mode, |\hskip| begins a new paragraph and puts glue after the
indentation.) \ If\/ |\indent| had been used instead, those lines
wouldn't have been indented any more than the first and third, because
|\indent| is implicit at the beginning of every paragraph.  Double
indentation on the second and fourth lines could have been achieved by
`|\indent\indent|'.''

\myhr

\noindent{\bf Summary}
\vskip1em
\noindent\item\bull Each item in a horizontal list is one of the following types of things

\halign{\hfil#\ &\ #\hfil\cr
a box                 & (a character or ligature or rule or hbox or vbox)\cr
a discretionary break & (to be explained momentarily)\cr
a ``whatsit''         & (something special to be explained later)\cr
vertical material     & (from |\mark| or |\vadjust| or |\insert|)\cr
a glob of glue        & (or |\leaders|, as we will see later)\cr
a kern                & (something like glue that doesn’t stretch or shrink)\cr
a penalty             & (representing the undesirability of breaking here)\cr
``math-on''           & (beginning a formula) or “math-off” (ending a formula)\cr
}
\vskip1em

\noindent\item\bull4 types {\it discardable}:
\halign{\hfil#\ &\ #\hfil\cr
non-discardable & box \cr
non-discardable & discretionary break \cr
non-discardable & ``whatsit'' \cr
non-discardable & vertical material \cr
discardable & glue\cr
discardable & kern\cr
discardable & penalty\cr
discardable & math items\cr
}

\myhr

\ansno14.8:
Official answer: ``|ba\ck/en| and |Be\ttt/uch|, where the macros |\ck/| and |\ttt/|
are defined by
\begintt
\def\ck/{\discretionary{k-}{k}{ck}}
\def\ttt/{tt\discretionary{-}{t}{}}
\endtt
The English word `eighteen' might deserve similar treatment.
\TeX's hyphenation algorithm will not make such spelling changes automatically.''

\ansno14.9:
Official answer: ``|\def\break{\penalty-10000 }|''

\ansno14.10:
Official answer: ``You get a forced break as if\/ |\nobreak| were not
present, because |\break| cannot be cancelled by another penalty. In
general if you have two penalties in a row, their combined effect is
the same as a single penalty whose value is the minimum of the two
original values, unless both of those values force breaks. \ (You get
two breaks from |\break\break|; the second one creates an empty
line.)''

\ansno14.11:
Official answer: ``Breaks are forced when $p\le-10000$, so there's no
point in subtracting a large constant whose effect on the total
demerits is known {\sl a priori}, especially when that might cause
arithmetic overflow.''

\ansno14.18:
\begintt
\newdimen\x
\x=1em
\setbox1=\hbox{I}
\setbox0=\vbox{\parshape=11 -0\x0\x -1\x2\x -2\x4\x -3\x6\x
   -4\x8\x -5\x10\x -6\x12\x -7\x14\x -8\x16\x -9\x18\x -10\x20\x
  \ifdim \x>2em \rightskip=-\wd1
  \else \frenchspacing \rightskip=-\wd1 plus1pt minus1pt
   \leftskip=0pt plus 1pt minus1pt \fi
  \parfillskip=0pt \tolerance=1000 \noindent
I turn, in the following treatises, to various uses of those triangles
whose generator is unity. But I leave out many more than I include; it
is extraordinary how fertile in properties this triangle is. Everyone
can try his hand.}
\centerline{\hbox to \wd1{\box0\hss}}
\endtt
\newdimen\x
\x=1em
\setbox1=\hbox{I}
\setbox0=\vbox{\parshape=11 -0\x0\x -1\x2\x -2\x4\x -3\x6\x
   -4\x8\x -5\x10\x -6\x12\x -7\x14\x -8\x16\x -9\x18\x -10\x20\x
  \ifdim \x>2em \rightskip=-\wd1
  \else \frenchspacing \rightskip=-\wd1 plus1pt minus1pt
   \leftskip=0pt plus 1pt minus1pt \fi
  \parfillskip=0pt \tolerance=1000 \noindent
I turn, in the following treatises, to various uses of those triangles
whose generator is unity. But I leave out many more than I include; it
is extraordinary how fertile in properties this triangle is. Everyone
can try his hand.}
\centerline{\hbox to \wd1{\box0\hss}}

\myhr

\noindent{\bf Summary}
\item\bull `|\item|' marco

\begintt
\item{1.} This is the first of several cases that are being
enumerated, with hanging indentation applied to entire paragraphs.
\itemitem{a)} This is the first subcase.
\itemitem{b)} And this is the second subcase. Notice
that subcases have twice as much hanging indentation.
\item{2.} The second case is similar.
\endtt
\borderedbox{
\item{1.} This is the first of several cases that are being
enumerated, with hanging indentation applied to entire paragraphs.
\itemitem{a)} This is the first subcase.
\itemitem{b)} And this is the second subcase. Notice
that subcases have twice as much hanging indentation.
\item{2.} The second case is similar.
}

\ansno14.19:
\begintt
\item{1.} a
\item{2.} b
\item{3.} c
\endtt

\item{1.} a
\item{2.} b
\item{3.} c

\ansno14.20:
|\item{$\bullet$}|
\item{$\bullet$} a

\ansno14.21:
{\let\endgraf=\par \edef\restorehsize{\hsize=\the\hsize}
\def\par{\endgraf \restorehsize \let\par=\endgraf}
\advance\hsize by-\parindent
\item\bull Lorem ipsum dolor sit amet, consectetur adipiscing elit. Proin dapibus lacus et quam hendrerit egestas. Sed gravida eros vitae felis condimentum gravida. Suspendisse efficitur tellus a euismod semper. Donec ac turpis pellentesque, commodo neque vitae, rhoncus massa.}

\ansno14.22:

\begintt
\dimen0=\hsize \advance\dimen0 by 2em
\parshape=3 0pt\hsize 0pt\hsize -2em\dimen0
\endtt

\noindent{Lorem ipsum dolor sit amet, consectetur adipiscing elit. Proin dapibus lacus et quam hendrerit egestas. Sed gravida eros vitae felis condimentum gravida. Suspendisse efficitur tellus a euismod semper. Donec ac turpis pellentesque, commodo neque vitae, rhoncus massa.}
\vskip1em
\noindent{\dimen0=\hsize \advance\dimen0 by 2em
\parshape=3 0pt\hsize 0pt\hsize -2em\dimen0
Lorem ipsum dolor sit amet, consectetur adipiscing elit. Proin dapibus lacus et quam hendrerit egestas. Sed gravida eros vitae felis condimentum gravida. Suspendisse efficitur tellus a euismod semper. Donec ac turpis pellentesque, commodo neque vitae, rhoncus massa.}
\vfill
\eject

\noindent Chapter 15
\myhr
\noindent{\bf Summary}

\ansno15.12:
\begintt
\newcount\notenumber
\def\clearnotenumber{\notenumber=0\relax}
\def\note{\advance\notenumber by 1
\footnote{$^{\the\notenumber}$}}
\endtt
\borderedbox{\def\clearnotenumber{\notenumber=0\relax}
\def\note{\advance\notenumber by 1
\footnote{$^{\the\notenumber}$}}
}
\myhr
\vfill
\eject


\noindent Chapter 16
\myhr
\ansno16.1:
`|$\gamma + \nu \in \Gamma$|' $\rightarrow$ `$\gamma + \nu \in \Gamma$'

\ansno16.2:
\halign{\indent\hfil#\ &\ #\hfil\cr
$\leq$ & |\leq|\cr
$\geq$ & |\geq|\cr
$\ne$ & |\ne|\cr
}

\myhr

\noindent{\bf Summary}
\item\bull when following a subformula it applies to that whole subformula, and
it will be raised or lowered accordingly
\beginmathdemo
|$((x^2)^3)^4$|&((x^2)^3)^4\cr
|${({(x^2)}^3)}^4$|&{({(x^2)}^3)}^4\cr
\endmathdemo

\myhr
\ansno16.3:
\halign{\indent\hfil#\ &\ #\ &\ #\hfil\cr
|$x + _2F_3$|   & $x + _2F_3$   & it applies to the plus sign \cr
|$x + {}_2F_3$| & $x + {}_2F_3$ & it applies to empty subformula \cr
}

\ansno16.4:
$F'(w,z)=\partial F(w,z)/\partial z$ and $F_\prime(w,z)=\partial F(w,z)/
\partial w$

\ansno16.6:
`$R_i{}^{jk}{}_l$' $\rightarrow$ `|$R_i{}^{jk}{}_l$|'

\ansno16.7:
\halign{\indent\hfil#\ &\ #\hfil\cr
$10^{10}$ & |$10^{10}$|\cr
$2^{n+1}$ & |$2^{n+1}$|\cr
$(n+1)^2$ & |$(n+1)^2$|\cr
$\sqrt{1-x^2}$ & |$\sqrt{1-x^2}$|\cr
$\overline{w+\overline{z}}$ & |$\overline{w+\overline{z}}$|\cr
$P_1^{e_1}$ & |$P_1^{e_1}$|\cr
$a_{b_{c_{d_e}}}$ & |$a_{b_{c_{d_e}}}$|\cr
$\root 3 \of {h''_n (\alpha x)}$ & |$\root 3 \of {h''_n (\alpha x)}$|\cr
}

\ansno16.8:
\halign{\indent\hfil#\ $\rightarrow$&\ #\hfil\cr
`If$ x = y$, then $x$ is equal to $y.$' & `If $x = y$, then $x$ is equal to $y$.'\cr
|If$ x = y$, then $x$ is equal to $y.$| & |If $x = y$, then $x$ is equal to $y$.|\cr
}

Offical answer: ``He got `If$ x = y\ldots$' because he forgot to leave
a space after `|If|'; ^{spaces} disappear between dollar signs. He
should also have ended the sentence with `|$y$.|'; punctuation that
belongs to a sentence should not be included in a formula, as we will
see in Chapter~18. \ (But you aren't expected to know that yet.)''

\ansno16.9:
`Deleting an element from an $n$-tuple leaves an $(n-1)$-tuple.'\parbreak
|`Deleting an element from an $n$-tuple leaves an $(n-1)$-tuple.'|

\ansno16.10:
Official answer: ``$Q,f,g,j,p,q,y$. \ (The analogous ^{Greek} letters
are ^^{italic letters with descenders} ^^{descenders}
$\beta,\gamma,\zeta,\eta,\mu,\xi,\rho,\phi,\varphi,\chi,\psi$.)''

\myhr
\noindent{\bf Summary}
\beginmathdemo
|$x\times y\cdot z$|&x\times y\cdot z\cr
|$x\circ y\bullet z$|&x\circ y\bullet z\cr
|$x\cup y\cap z$|&x\cup y\cap z\cr
|$x\sqcup y\sqcap z$|&x\sqcup y\sqcap z\cr
|$x\vee y\wedge z$|&x\vee y\wedge z\cr
|$x\pm y\mp z$|&x\pm y\mp z\cr
|$x=+1$|&x=+1\cr
|$3.142-$|&3.142-\cr
|$(D*)$|&(D*)\cr
\endmathdemo
\myhr

\ansno16.11:
\halign{\indent\hfil#\ &\ #\hfil\cr
|$z^{*2}$|  &$z^{*2}$\cr
|$h_*'(z)$| & $h_*'(z)$\cr
}
\myhr
\noindent{\bf Summary}
\beginmathdemo
|$x=y>z$|&x=y>z\cr
|$x:=y$|&x:=y\cr
|$x\le y\ne z$|&x\le y\ne z\cr
|$x\sim y\simeq z$|&x\sim y\simeq z\cr
|$x\equiv y\not\equiv z$|&x\equiv y\not\equiv z\cr
|$x\subset y\subseteq z$|&x\subset y\subseteq z\cr
|$f:A\to B$|&f:A\to B\cr
|$f\colon A\to B$|&f\colon A\to B\cr
|$12,345x$|&12,345x\qquad\rm(wrong)\cr
|$12{,}345x$|&12{,}345x\qquad\,\rm(right)\cr
\endmathdemo
\myhr

\ansno16.12:
`|$3{\cdot}1416$|' $\rightarrow$ `$3{\cdot}1416$'

\myhr
\noindent{\bf Summary}
\beginmathdemo
|$\hat a$|&\hat a\cr
|$\check a$|&\check a\cr
|$\tilde a$|&\tilde a\cr
|$\acute a$|&\acute a\cr
|$\grave a$|&\grave a\cr
|$\dot a$|&\dot a\cr
|$\ddot a$|&\ddot a\cr
|$\breve a$|&\breve a\cr
|$\bar a$|&\bar a\cr
|$\vec a$|&\vec a\cr
|$\widehat x,\widetilde x$|&\tenmath\widehat x,\widetilde x\cr
|$\widehat{xy},\widetilde{xy}$|&\tenmath\widehat{xy},\widetilde{xy}\cr
|$\widehat{xyz},\widetilde{xyz}$|&\tenmath\widehat{xyz},\widetilde{xyz}\cr
\endmathdemo
\myhr
\ansno16.13:
{\def\ghat{{\hat g}}
\beginmathdemo
|$e^{-x^2}$| & e^{-x^2}\cr
|$D\sim p^\alpha M+l$| & D\sim p^\alpha M+l\cr
|$\ghat\in(H^{\pi_1^{-1}})'$| & \ghat\in(H^{\pi_1^{-1}})'\cr
\endmathdemo}
\vfill
\eject

\noindent Chapter 17.

\myhr

\noindent{\bf Summary}
\begindisplaymathdemo
\it Input&\it Output\cr
\noalign{\vskip-3pt}
|$$x+y^2\over k+1$$|&x+y^2\over k+1\cr
\noalign{\vskip2pt}
|$${x+y^2\over k}+1$$|&{x+y^2\over k}+1\cr
\noalign{\vskip-1pt}
|$$x+{y^2\over k}+1$$|&x+{y^2\over k}+1\cr
\noalign{\vskip-1pt}
|$$x+{y^2\over k+1}$$|&x+{y^2\over k+1}\cr
\noalign{\vskip-3pt}
|$$x+y^{2\over k+1}$$|&x+y^{2\over k+1}\cr
\endmathdemo

\myhr

\ansno17.1:
\beginmathdemo
|$x+y^{2/(k+1)}$|&x+y^{2/(k+1)}\cr
\endmathdemo

\ansno17.2:
\beginmathdemo
|$((a+1)/(b+1))x$|&((a+1)/(b+1))x\cr
\endmathdemo

\ansno17.3:
Official answer: ``He got the displayed formula$$x=(y^2\over k+1)$$ because he forgot
that an unconfined |\over| applies to everything.  \ (He should probably
have typed `|$$x=\left(y^2\over k+1\right)$$|', using ideas that will be
presented later in this chapter; this not only makes the parentheses
larger, it keeps the `$x=$' out of the fraction, because |\left| and
|\right| introduce subformulas.)''\parbreak
Correct version: $$x=\left(y^2\over k+1\right)$$

\ansno17.4:
{\def\cents{\hbox{\rm\rlap/c}}
\beginmathdemo
|$7{1\over2}\cents$| & 7{1\over2}\cents\cr
or |7$1\over2$\cents| & \cr
\endmathdemo}

\ansno17.5:
Official answer: ``Style $D'$ is used for the subformula $p_2^{e'}$, hence style~$S'$
is used for the superscript~$e'$ and the subscript~2, and style~$\SS'$
is used for the supersuperscript prime. The square root sign and the $p$
appear in text size; the 2 and the~$e$ appear in script size; and the
$\prime$ is in scriptscript size.''

\myhr

\noindent{\bf Summary}
\item\bull continued fraction (with |\strut| and |\displaystyle|)
$$a_0+{1\over\displaystyle a_1+
          {\strut 1\over\displaystyle a_2+
            {\strut 1\over\displaystyle a_3+
              {\strut 1\over a_4}}}}$$
\item\bull continued fraction (without |\strut| and |\displaystyle|)
$$a_0+{1\over a_1+{1\over
      a_2+{1\over a_3+{1\over a_4}}}}$$
\item\bull continued fraction (with flush left)
$$a_0+{1\hfill\over\displaystyle a_1+
          {\strut1\hfill\over\displaystyle a_2+
            {\strut1\hfill\over\displaystyle a_3+
              {\strut1\over a_4}}}}$$
\item\bull |\atop|
\begindisplaymathdemo
|$$x\atop y+2$$|&x\atop y+2\cr
\endmathdemo

\item\bull |\choose|
\begindisplaymathdemo
|$$n\choose k$$|&n\choose k\cr
\endmathdemo

\myhr

\ansno17.6:
\item\bull |$${1\over2}{n\choose k}$$| $${1\over2}{n\choose k}$$
\item\bull |$$\displaystyle{n\choose k}\over2$$| $$\displaystyle{n\choose k}\over2$$

\ansno17.7:
\item\bull |$${p \choose 2} x^2 y^{p-2} - {1 \over 1-x}{1 \over 1-x^2}.$$|
$${p \choose 2} x^2 y^{p-2} - {1 \over 1-x}{1 \over 1-x^2}.$$

\myhr
\noindent{\bf Summary}
\item\bull |\sum|
$$\halign{\indent#\hfil\qquad yields\qquad&$#\hfil$\qquad&#\hfil\cr
|$\sum x_n$|&\sum x_n&($T$ style)\cr
\noalign{\vskip3pt}
|$$\sum x_n$$|&\displaystyle\sum x_n&($D$ style).\cr}$$
\item\bull |\infty|
$$\halign{\indent\hbox to2.3in{#\hfil}\hbox to.6in{yields\hfil}&
  $#\hfil$\qquad&#\hfil\cr
|$\int_{-\infty}^{+\infty}$|&\int_{-\infty}^{+\infty}&($T$ style)\cr
\noalign{\vskip3pt}
|$$\int_{-\infty}^{+\infty}$$|&\displaystyle\int_{-\infty}^{+\infty}&
  ($D$ style).\cr}$$
\item\bull |\atop| in |\sum|
\begintt
$$\sum_{\scriptstyle0\le i\le m\atop\scriptstyle0<j<n}P(i,j)$$
\endtt
$$\sum_{\scriptstyle0\le i\le m\atop\scriptstyle0<j<n}P(i,j)$$

\myhr
\ansno17.8:
|$$\sum_{i=1}^p\sum_{j=1}^q\sum_{k=1}^ra_{ij}b_{jk}c_{ki}$$|
$$\sum_{i=1}^p\sum_{j=1}^q\sum_{k=1}^ra_{ij}b_{jk}c_{ki}$$

\ansno17.9:
|$$\sum_{{\scriptstyle 1\le i\le p \atop \scriptstyle 1\le j\le q}
    \atop \scriptstyle 1\le k\le r} a_{ij} b_{jk} c_{ki}$$|
$$\sum_{{\scriptstyle 1\le i\le p \atop \scriptstyle 1\le j\le q}
    \atop \scriptstyle 1\le k\le r} a_{ij} b_{jk} c_{ki}$$

\myhr
\noindent{\bf Summary}
\item\bull groups of square roots
\begintt
$$\sqrt{1+\sqrt{1+\sqrt{1+
            \sqrt{1+\sqrt{1+\sqrt{1+\sqrt{1+x}}}}}}}$$
\endtt
$$\sqrt{1+\sqrt{1+\sqrt{1+
            \sqrt{1+\sqrt{1+\sqrt{1+\sqrt{1+x}}}}}}}$$
\item\bull delimiters
\begindisplay
\it Input         & \it Delimiter\cr
\noalign{\vskip2pt}
|(|               & left parenthesis: $($\cr
|)|               & right parenthesis: $)$\cr
|[| or |\lbrack|  & left bracket: $[$\cr
|]| or |\rbrack|  & right bracket: $]$\cr
|\{| or |\lbrace| & left curly brace: $\{$\cr
|\}| or |\rbrace| & right curly brace: $\}$\cr
|\lfloor|         & left floor bracket: $\lfloor$\cr
|\rfloor|         & right floor bracket: $\rfloor$\cr
|\lceil|          & left ceiling bracket: $\lceil$\cr
|\rceil|          & right ceiling bracket: $\rceil$\cr
|\langle|         & left angle bracket: $\langle$\cr
|\rangle|         & right angle bracket: $\rangle$\cr
|/|               & slash: $/$\cr
|\backslash|      & reverse slash: $\backslash$\cr
\| or |\vert|     & vertical bar: $\vert$\cr
|\|\| or |\Vert|  & double vertical bar: $\Vert$\cr
|\uparrow|        & upward arrow: $\uparrow$\cr
|\Uparrow|        & double upward arrow: $\Uparrow$\cr
|\downarrow|      & downward arrow: $\downarrow$\cr
|\Downarrow|      & double downward arrow: $\Downarrow$\cr
|\updownarrow|    & up-and-down arrow: $\updownarrow$\cr
|\Updownarrow|    & double up-and-down arrow: $\Updownarrow$\cr
\enddisplay

\item\bull |\bigl|, |\bigr|
\beginlongmathdemo
\it Input&\it Output\cr
\noalign{\vskip2pt}
|$\bigl(x-s(x)\bigr)\bigl(y-s(y)\bigr)$|&
  \bigl(x-s(x)\bigr)\bigl(y-s(y)\bigr)\cr
|$\bigl[x-s[x]\bigr]\bigl[y-s[y]\bigr]$|&
  \bigl[x-s[x]\bigr]\bigl[y-s[y]\bigr]\cr
|$\bigl|\|| |\||x|\||-|\||y|\|| \bigr|\||$|&
  \bigl\vert\vert x\vert-\vert y\vert\bigr\vert\cr
|$\bigl\lfloor\sqrt A\bigr\rfloor$|&
  \bigl\lfloor\sqrt A\bigr\rfloor\cr
\endmathdemo
\item\bull |\Bigl|, |\Bigr|
\item\bull |\biggl|, |\biggr|
\item\bull |\Biggl|, |\Biggr|

\myhr

\ansno17.10:
|$\displaystyle\biggl({\partial^2\over\partial x^2}+|\hfil\break
|{\partial^2\over\partial y^2}\biggr)\bigl|\||\varphi(x+iy)\bigr|\||^2=0$|.\parbreak
$\displaystyle\biggl({\partial^2\over\partial x^2}+
{\partial^2\over\partial y^2}\biggr)\bigl\vert\varphi(x+iy)\bigr\vert^2=0$

\ansno17.11:
Official answer: ``Formulas that are more than one line tall are
usually two lines tall, not 1$1\over2$ or 2$1\over2$ lines tall.''

\ansno17.12:
|$\bigl(x+f(x)\bigr) \big/ \bigl(x-f(x)\bigr)$|
$\bigl(x+f(x)\bigr) \big/ \bigl(x-f(x)\bigr)$

\ansno17.13:
|$$\pi(n)=\sum_{k=2}^n\left\lfloor\phi(k)\over k-1\right\rfloor.$$|
$$\pi(n)=\sum_{k=2}^n\left\lfloor\phi(k)\over k-1\right\rfloor.$$

\ansno17.14:
|$$\pi(n)=\sum_{m=2}^n\left\lfloor\biggl(\sum_{k=1}^{m-1}\bigl|
\hfil\break
|\lfloor(m/k)\big/\lceil m/k\rceil\bigr\rfloor\biggr)^{-1}\right\rfloor.$$|
$$\pi(n)=\sum_{m=2}^n\left\lfloor\biggl(\sum_{k=1}^{m-1}\bigl\lfloor
  (m/k)\big/\lceil m/k\rceil\bigr\rfloor\biggr)^{-1}\right\rfloor.$$

\ansno17.15:
\begintt
\def\puzzle{{\mathchoice{D}{T}{S}{SS}}}
$$\puzzle{\puzzle\over\puzzle^{\puzzle^\puzzle}}$$
\endtt
{\def\puzzle{{\mathchoice{D}{T}{S}{SS}}}
$$\puzzle{\puzzle\over\puzzle^{\puzzle^\puzzle}}$$}
$\equiv$ |$${D}{{T}\over{T}^{{S}^{SS}}}$$|
$${D}{{T}\over{T}^{{S}^{SS}}}$$

\ansno17.16:
|\def\sqr#1#2{{\vcenter{\vbox{\hrule height.#2pt          |\parbreak
|        \hbox{\vrule width.#2pt height#1pt \kern#1pt     |\parbreak
|           \vrule width.#2pt}                            |\parbreak
|        \hrule height.#2pt}}}}                           |\parbreak
|\def\square{\mathchoice\sqr34\sqr34\sqr{2.1}3\sqr{1.5}3} |

{\def\sqr#1#2{{\vcenter{\vbox{\hrule height.#2pt
        \hbox{\vrule width.#2pt height#1pt \kern#1pt
           \vrule width.#2pt}
        \hrule height.#2pt}}}}
\def\square{\mathchoice\sqr34\sqr34\sqr{2.1}3\sqr{1.5}3}
$\square$}

\ansno17.17:
{|\def\euler{\atopwithdelims<>}|
\def\euler{\atopwithdelims<>}
$$n\euler k$$}

\ansno17.18:
Official answer: ``The |\textfont0| that was current at the beginning of the formula
will be used, because this redefinition is local to the braces. \
(It would be a different story if `^|\global||\textfont|' had appeared instead;
that would have changed the meaning of\/ |\textfont0| at all levels.)''

\ansno17.19:
Official answer: ``\hex{2208} and \hex{220F}.''
\vfill
\eject


\noindent Chapter 18.
\myhr
\ansno18.1:
|$R(n,t)=O(t^{n/2})$, as $t\to0^+$.|
$R(n,t)=O(t^{n/2})$, as $t\to0^+$.

\ansno18.2:
|$$p_1(n) = \lim_{m\to\infty}\sum_{\nu=0}^\infty|\parbreak
|\left(1-\cos^{2m}(\nu!^n\pi/n)\right)$$|
$$p_1(n) = \lim_{m\to\infty}\sum_{\nu=0}^\infty\left(1-\cos^{2m}(\nu!^n\pi/n)\right)$$

\ansno18.3:
\begintt
\def\limsup{\mathop{\overline{\rm lim}}}
\def\liminf{\mathop{\underline{\rm lim}}}
$$\lim_{n\to\infty}x_n{\rm\ exists}\iff
  \limsup_{n\to\infty}x_n=\liminf_{n\to\infty}x_n.$$
\endtt
{\def\limsup{\mathop{\overline{\rm lim}}}
\def\liminf{\mathop{\underline{\rm lim}}}
$$\lim_{n\to\infty}x_n{\rm\ exists}\iff
  \limsup_{n\to\infty}x_n=\liminf_{n\to\infty}x_n.$$}

\myhr
\noindent{\bf Summary}
\item\bull |\bmod| (binary mod), |\pmod| (parenthetical mod)
\beginmathdemo
|$\gcd(m,n)=\gcd(n,m\bmod n)$|&\gcd(m,n)=\gcd(n,m\bmod n)\cr
|$x\equiv y+1\pmod{m^2}$|&x\equiv y+1\pmod{m^2}\cr
\endmathdemo

\myhr

\ansno18.4:
$x\equiv0(\pmod y^n)$. He should have typed
`|$x\equiv0\pmod{y^n}$|'. $x\equiv0\pmod{y^n}$

\ansno18.5:
|$${n\choose k}\equiv{\lfloor n/p\rfloor\choose|\parbreak
|  \lfloor k/p\rfloor}{n\bmod p\choose k\bmod p}\pmod p.$$|
$${n\choose k}\equiv{\lfloor n/p\rfloor\choose
  \lfloor k/p\rfloor}{n\bmod p\choose k\bmod p}\pmod p.$$

\ansno18.6:
|$\bf\bar x^{\rm T}Mx={\rm0}\iff x=0$|\parbreak
$\bf\bar x^{\rm T}Mx={\rm0}\iff x=0$

\ansno18.7:
|$S\subseteq{\mit\Sigma}\iff S\in{\cal S}$|\parbreak
$S\subseteq{\mit\Sigma}\iff S\in{\cal S}$

\ansno18.8:
|$${\it available}+\sum_{i=1}^n\max\bigl({\it full}(i),|\parbreak
|{\it reserved}(i)\bigr)={\it capacity}.$$|
$$
{\it available}+\sum_{i=1}^n\max\bigl({\it full}(i),{\it reserved}(i)\bigr)
  ={\it capacity}.$$

\ansno18.9:
\begintt
$$\vbox{\let\par=\endgraf
\obeylines\sfcode`;=3000
{\bf for $j:=2$ step $1$ until $n$ do}
\quad {\bf begin} ${\it accum}:=A[j]$; $k:=j-1$; $A[0]:=\it accum$;
\quad {\bf while $A[k]>\it accum$ do}
\qquad {\bf begin} $A[k+1]:=A[k]$; $k:=k-1$;
\qquad {\bf end};
\quad $A[k+1]:=\it accum$;
\quad {\bf end}.
}$$
\endtt

$$\vbox{\let\par=\endgraf
\obeylines\sfcode`;=3000
{\bf for $j:=2$ step $1$ until $n$ do}
\quad {\bf begin} ${\it accum}:=A[j]$; $k:=j-1$; $A[0]:=\it accum$;
\quad {\bf while $A[k]>\it accum$ do}
\qquad {\bf begin} $A[k+1]:=A[k]$; $k:=k-1$;
\qquad {\bf end};
\quad $A[k+1]:=\it accum$;
\quad {\bf end}.
}$$

\ansno18.10:
|Let $H$~be a Hilbert space,                 |\parbreak
|$C$~a closed bounded convex subset of~$H$,  |\parbreak
|$T$~a nonexpansive self map of~$C$.         |\parbreak
|Suppose that as $n\to\infty$,               |\parbreak
|$a_{n,k}\to0$ for each~$k$,  |\parbreak
|and $\gamma_n=\sum_{k=0}^\infty(a_{n,k+1}-|\allowbreak|a_{n,k})^+\to0$.  |\parbreak
| Then for each $x$~in~$C$,                                               |\parbreak
|$A_nx=\sum_{k=0}^\infty a_{n,k}T^kx$ converges weakly                    |\parbreak
|to a fixed point of~$T$.                                                 |\parbreak

Let $H$~be a Hilbert space,                                 \
$C$~a closed bounded convex subset of~$H$,                  \
$T$~a nonexpansive self map of~$C$.                         \
Suppose that as $n\to\infty$,                               \
$a_{n,k}\to0$ for each~$k$,                                 \
and $\gamma_n=\sum_{k=0}^\infty(a_{n,k+1}-a_{n,k})^+\to0$.  \
 Then for each $x$~in~$C$,                                  \
$A_nx=\sum_{k=0}^\infty a_{n,k}T^kx$ converges weakly       \
to a fixed point of~$T$.

\myhr
\noindent{\bf Summary}
\item\bull thin space, medium space, thick space, negative space
$$\halign{\indent#\hfil&\quad#\hfil\cr
|\,|&thin space \ (normally 1/6 of a quad);\cr
|\>|&medium space \ (normally 2/9 of a quad);\cr
|\;|&thick space \ (normally 5/18 of a quad);\cr
|\!|&negative thin space \ (normally $-1/6$ of a quad).\cr}$$

\item\bull thin space: formula, before $dx$, $dy$

\beginmathdemo
\it Input&\it Output\cr
\noalign{\vskip2pt}
|$\int_0^\infty f(x)\,dx$|&\int_0^\infty f(x)\,dx\cr
|$y\,dx-x\,dy$|&y\,dx-x\,dy\cr
|$dx\,dy=r\,dr\,d\theta$|&dx\,dy=r\,dr\,d\theta\cr
|$x\,dy/dx$|&x\,dy/dx\cr \endmathdemo Notice that no `|\,|' was desirable
after the `|/|' in the last example.  Similarly, there's no need for
`|\,|' in cases like
\begindisplaymathdemo
|$$\int_1^x{dt\over t}$$|&\int_1^x{dt\over t}\cr
\endmathdemo

\item\bull thin space: Expression with physical unit

When physical units appear in a formula, they should be set in roman
type and separated from the preceding material by a thin space:
\beginmathdemo
|$55\rm\,mi/hr$|&55\rm\,mi/hr\cr
|$g=9.8\rm\,m/sec^2$|&g=9.8\rm\,m/sec^2\cr
|$\rm1\,ml=1.000028\,cc$|&\rm1\,ml=1.000028\,cc\cr
\endmathdemo

\item\bull thin space: after exclamation points
\beginmathdemo
|$(2n)!/\bigl(n!\,(n+1)!\bigr)$|&(2n)!/\bigl(n!\,(n+1)!\bigr)\cr
\noalign{\vskip6pt}
|$${52!\over13!\,13!\,26!}$$|&\displaystyle{52!\over13!\,13!\,26!}\cr
\endmathdemo

\item\bull Use |\,| or |\!| to adjust spacing for different cases
\beginmathdemo
|$\sqrt2\,x$|&\sqrt2\,x\cr
|$\sqrt{\,\log x}$|&\sqrt{\,\log x}\cr
|$O\bigl(1/\sqrt n\,\bigr)$|&O\bigl(1/\sqrt n\,\bigr)\cr
|$[\,0,1)$|&[\,0,1)\cr
|$\log n\,(\log\log n)^2$|&\log n\,(\log\log n)^2\cr
|$x^2\!/2$|&x^2\!/2\cr
|$n/\!\log n$|&n/\!\log n\cr
|$\Gamma_{\!2}+\Delta^{\!2}$|&\Gamma_{\!2}+\Delta^{\!2}\cr
|$R_i{}^j{}_{\!kl}$|&R_i{}^j{}_{\!kl}\cr
|$\int_0^x\!\int_0^y dF(u,v)$|&\int_0^x\!\int_0^y dF(u,v)\cr
\noalign{\vskip6pt}
|$$\int\!\!\!\int_D dx\,dy$$|&\displaystyle{\int\!\!\!\int_D dx\,dy}\cr
\endmathdemo


\myhr
\ansno18.11:
|$$\int_0^\infty{t-ib\over t^2+b^2}e^{iat}\,dt=|\parbreak
|    e^{ab}E_1(ab),\qquad a,b>0.$$|
$$\int_0^\infty {t-ib \over t^2+b^2}e^{iat}\,dt=e^{ab}E_1(ab),\qquad a,b>0.$$

\ansno18.12:
|$$\hbar=1.0545\times10^{-27}\rm\,erg\,sec.$$|
$$\hbar=1.0545\times10^{-27}\rm\,erg\,sec.$$

\myhr
\noindent{\bf Summary}
\item\bull eight basic types in math list
$$\baselineskip0pt\lineskip0pt
\halign to\hsize
 {\strut\hbox to\parindent{\it#\hfil}& % for the legend "Left atom"
  #\hfil\quad& % for the row labels
  #\hfil\tabskip 0pt plus 10pt& % for the rule at the left
  \hbox to 25pt{\tt\hss#\hss}& % for column 1
  \hbox to 25pt{\tt\hss#\hss}& % for column 2
  \hbox to 25pt{\tt\hss#\hss}& % for column 3
  \hbox to 25pt{\tt\hss#\hss}& % for column 4
  \hbox to 25pt{\tt\hss#\hss}& % for column 5
  \hbox to 25pt{\tt\hss#\hss}& % for column 6
  \hbox to 25pt{\tt\hss#\hss}& % for column 7
  \hbox to 25pt{\tt\hss#\hss}& % for column 8
  #\hfil\tabskip0pt\cr % for the rule at the right
&&&&\multispan7\hss\it Right atom\hss\cr
\noalign{\vskip3pt}
&&&\rm Ord&\rm Op&\rm Bin&\rm Rel&\rm Open&\rm Close&\rm Punct&\rm Inner\cr
\noalign{\vskip2pt}
\omit&&\multispan{10}\leaders\hrule\hfil\cr
\omit\vbox to 2pt{}&&\vrule&&&&&&&&&\vrule\cr
&Ord&\vrule&0&1&(2)&(3)&0&0&0&(1)&\vrule\cr
&Op&\vrule&1&1&*&(3)&0&0&0&(1)&\vrule\cr
&Bin&\vrule&(2)&(2)&*&*&(2)&*&*&(2)&\vrule\cr
Left&Rel&\vrule&(3)&(3)&*&0&(3)&0&0&(3)&\vrule\cr
atom&Open&\vrule&0&0&*&0&0&0&0&0&\vrule\cr
&Close&\vrule&0&1&(2)&(3)&0&0&0&(1)&\vrule\cr
&Punct&\vrule&(1)&(1)&*&(1)&(1)&(1)&(1)&(1)&\vrule\cr
&Inner&\vrule&(1)&1&(2)&(3)&(1)&0&(1)&(1)&\vrule\cr
\omit\vbox to 2pt{}&&\vrule&&&&&&&&&\vrule\cr
\omit&&\multispan{10}\leaders\hrule\hfil\cr}$$
\myhr

\ansno18.13:
\begindisplay \def\0{\thinspace}%
  \def\1{\thinspace{\tt\bslash,}\thinspace}%
  \def\2{\thinspace{\tt\bslash>}\thinspace}%
  \def\3{\thinspace{\tt\bslash;}\thinspace}
Ord\0Open\0Ord\0Punct\1Ord\0Close\3Rel\3Ord\2Bin\2Ord.
\enddisplay

\ansno18.14:
|$\left]-\infty,T\right[\times\left]-\infty,T\right[$|
$\left]-\infty,T\right[\times\left]-\infty,T\right[$

\ansno18.15:
Official answer: ``The first |+| will become a Bin atom, the second an
Ord; hence the result is $x$, medium space, $+$, medium space, $+$, no
space, 1.''

\myhr
\noindent{\bf Summary}
\item\bull ellipses
\item\item\bull\quad correct
\begindisplay
$\displaystyle x_1+\cdots+x_n\qquad {\rm and}\qquad (x_1,\ldots,x_n),$
\enddisplay
\item\item\bull\quad wrong
\begindisplay
$\displaystyle x_1+\ldots+x_n\qquad {\rm and}\qquad (x_1,\cdots,x_n).$
\enddisplay
\item\bull other cases
\beginmathdemo
|$x_1+\cdots+x_n$|&x_1+\cdots+x_n\cr
|$x_1=\cdots=x_n=0$|&x_1=\cdots=x_n=0\cr
|$A_1\times\cdots\times A_n$|&A_1\times\cdots\times A_n\cr
|$f(x_1,\ldots,x_n)$|&f(x_1,\ldots,x_n)\cr
|$x_1x_2\ldots x_n$|&x_1x_2\ldots x_n\cr
|$(1-x)(1-x^2)\ldots(1-x^n)$|&(1-x)(1-x^2)\ldots(1-x^n)\cr
|$n(n-1)\ldots(1)$|&n(n-1)\ldots(1)\cr
\endmathdemo
\item\bull at end of formula, thin space (|\,|, |~|) is needed
\item\item\bull |Prove that $(1-x)^{-1}=1+x+x^2+\cdots\,$.|\break
Prove that $(1-x)^{-1}=1+x+x^2+\cdots\,$.
\item\item\bull |Clearly $a_i<b_i$ for $i=1$,~2, $\ldots\,$,~$n$.|\break
Clearly $a_i<b_i$ for $i=1$,~2, $\ldots\,$,~$n$.
\item\item\bull |The coefficients $c_0$,~$c_1$, \dots,~$c_n$ are positive.|\break
The coefficients $c_0$,~$c_1$, \dots,~$c_n$ are positive.

\myhr
\ansno18.16:
\item\bull |x_1+x_1x_2+\cdots+x_1x_2|\break
$x_1+x_1x_2+\cdots+x_1x_2$

\item\bull |(x_1,\ldots,x_n)\cdot(y_1,\ldots,y_n)=x_1y_1+\cdots+x_ny_n|\break
$(x_1,\ldots,x_n)\cdot(y_1,\ldots,y_n)=x_1y_1+\cdots+x_ny_n$

\ansno18.17:
Official answer: ``The commas belong to the sentence, not to the
formula; his decision to put them into math mode meant that \TeX\
didn't put large enough spaces after them. Also, his formula `$i=1,
2, \ldots, n$' allows no breaks between lines, except after the $=$,
so he's risking overfull box problems. But suppose the sentence had
been more terse:
\begindisplay
Clearly $a_i<b_i$ \ ($i=1,2,\ldots,n$).
\enddisplay
Then his idea would be basically correct:
\begintt
Clearly $a_i<b_i$ \ ($i=1,2,\ldots,n$).
\endtt''

\ansno18.18:
Official answer: ``$\ldots$ |never\footnote*{Well \dots, hardly ever.}
have| $\ldots$''

\ansno18.19:
Offical answer: ``Neither formula will be broken between lines, but
the thick spaces in the second formula will be set to their natural
width while the thick spaces in the first formula will retain their
stretchability.''

\ansno18.20:
Official answer: ``Set |\relpenalty||=10000| and
|\binoppenalty||=10000|.  And you also need to change the definitions
of\/ |\bmod| and |\pmod|, which insert their own penalties.''

\ansno18.21:
|$$\bigl\{\,x^3\bigm|\||h(x)\in\{-1,0,+1\}\,\bigr\}$$|\parbreak
$$\bigl\{\,x^3\bigm\vert h(x)\in\{-1,0,+1\}\,\bigr\}$$

\ansno18.22:
Official answer: ``|$\{\,p\mid p$~and $p+2$ are prime$\,\}$|, assuming that
^|\mathsurround| is zero. The more difficult alternative
`|$\{\,p\mid p\ {\rm and}\ p+2\rm\ are\ prime\,\}$|' is not a solution,
because line breaks do not occur at |\|\] ^^|\space|
(or at glue of any kind) within math formulas. Of course it may be best to
display a formula like this, instead of breaking it between lines.''

\ansno18.23:
|$$f(x)=\cases{1/3&if $0\le x\le1$;\cr 2/3&if $3\le x\le4$;\cr|\hfil
\break|0&elsewhere.\cr}$$|
$$f(x) = \cases{1/3, &if $0 \leq x \leq 1$;\cr
2/3, &if $3 \leq x \leq 4$;\cr
0, &elsewhere\cr}$$

\myhr
\noindent{\bf Summary}
\item\bull cases
\item\bull horizontal braces
\beginlongdisplaymathdemo
\noalign{\vskip9pt}
|$$\overbrace{x+\cdots+x}^{k\rm\;times}$$|&
  \overbrace{x+\cdots+x}^{k\rm\;times}\cr
\noalign{\vskip-6pt}
|$$\underbrace{x+y+z}_{>\,0}.$$|&
  \underbrace{x+y+z}_{>\,0}.\cr
\endmathdemo
\item\bull matrices
\begintt
$$A=\left(\matrix{x-\lambda&1&0\cr
                  0&x-\lambda&1\cr
                  0&0&x-\lambda\cr}\right).$$
\endtt
$$A=\left(\matrix{x-\lambda&1&0\cr
                  0&x-\lambda&1\cr
                  0&0&x-\lambda\cr}\right).$$
\item\bull |\pmatrix|

\ansno18.24:
|$$\left\lgroup\matrix{a&b&c\cr d&e&f\cr}\right\rgroup$$|\parbreak
$$\left\lgroup\matrix{a&b&c\cr d&e&f\cr}\right\rgroup$$
|$$\left\lgroup\matrix{u&x\cr v&y\cr w&z\cr}\right\rgroup$$|\parbreak
$$\left\lgroup\matrix{u&x\cr v&y\cr w&z\cr}\right\rgroup$$


%% [2025-05-22T07:09:46+08:00] display \vsize and \hsize
%% \message{The current \string\vsize\space is \the\vsize.}
%% \message{The current \string\hsize\space is \the\hsize.}
\bye
