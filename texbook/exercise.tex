\input manmac

\noindent Chapter 1
\vskip\baselineskip\hrule\vskip.5\baselineskip
\ansno1.1:
 A \TeX nician (underpaid); sometimes also called a \TeX acker.
\vskip\baselineskip\hrule\vskip.5\baselineskip
\vfill
\eject


\noindent Chapter 2

\vskip\baselineskip\hrule\vskip.5\baselineskip
%% [2025-05-06T18:14:46+08:00]
``I understand''

%% [2025-05-06T18:12:00+08:00] Do not write comment inside the block
%% between `\begintt' and `\endtt' because the anything between
%% `\begintt' and `\endtt' will be eventually printed out in output
%% pdf.

%% [2025-05-06T18:16:22+08:00] `\]' represents "visible spacebar" in
%% normal/display mode
Visible spacebar: ``I\]understand.''

%% [2025-05-06T18:13:44+08:00] `|]' represents "visible spacebar" in
%% verbatim mode
\begintt
Verbatim spacebar: ``I|]understand.''
\endtt

\begindisplay
|-| & hyphen (-)\cr
|--| & en-dash (--)\cr
|---| & em-dash (---)\cr
|$-$| & minus sign $-$\cr
\enddisplay
\hrule

\ansno2.1: Alice said, ``I always use an en-dash instead of a hyphen
when specifying page numbers like `480--491' in a bibliography.''


\ansno2.2: Four hyphens `|----|' becomes one em-dash and one hyphen
`----'

\ansno2.3: fluff

\vskip\baselineskip\hrule\vskip.5\baselineskip
Summary: ligatures, kerning


alternative input methods for left quotes, right quotes
\begindisplay
{\tt |\lq\lq\]I\]understand.\rq\rq|} $\Rightarrow$ \lq\lq\]I\]understand.\rq\rq
\enddisplay
\hrule


\ansno2.4: ``\thinspace` and `\thinspace``


\ansno2.5: Following Occam's Razor Principle, introducing extra
complexity on typewriting annotations might not be a wise move because
it increases extra maintenance cost in minds and memory. Besides,
introducing dollar sign might cause confusions between {\sl Math mode}
and {\sl Text mode} that it may deteriorate maintenability of code.

\vfill
\eject


\noindent Chapter 3.
\vskip\baselineskip\hrule\vskip.5\baselineskip

\vfill
\eject


\end
